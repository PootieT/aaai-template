%%%%%%%%%%%%%%%%%%%%%%%%%%%%%%%%%%%%%%%%%
% Short Stylish Cover Letter
% LaTeX Template
% Version 1.0 (28/5/13)
%
% This template has been downloaded from:
% http://www.LaTeXTemplates.com
%
% Original author:
% Stefano (http://stefano.italians.nl/archives/63)
%
% License:
% CC BY-NC-SA 3.0 (http://creativecommons.org/licenses/by-nc-sa/3.0/)
%
%%%%%%%%%%%%%%%%%%%%%%%%%%%%%%%%%%%%%%%%%

\documentclass[12pt]{letter}

\usepackage{marvosym}
\usepackage{helvet}
\usepackage{times}
\usepackage{courier}
\usepackage{xspace}
\usepackage{hyperref}
\usepackage[T1]{fontenc}
% \usepackage{mycv}
\usepackage[top=27mm,left=27mm,right=27mm,bottom=27mm]{geometry}

\pagestyle{empty}

\begin{document}

{
\large
Response to the selection criteria for the Post-doc position (Job ID 516745)\\
}

Dear Dr. Patrik Haslum,

\vspace{1em}

\setlength{\parskip}{0.3em}
% PARAGRAPH ONE: State the reason for the letter, name the position or
 % type of work you are applying for and identify the source from which
 % you learned of the opening.\\ 

% PARAGRAPH TWO: Indicate why you are interested in the position, the
% company, its products, services - above all, stress what you can do
% for the employer. If you are a recent graduate, explain how your
% academic background makes you a qualified candidate for the
% position. If you have practical work experience, point out specific
% achievements or unique qualifications. Try not to repeat the same
% information the reader will find in the resume. The purpose of this
% section is to strengthen your resume by providing details which bring
% your experiences to life.\\ 


% Please find enclosed CV in the application.

I am going to receive a Ph.D degree from the current institution on March 2018.
I have strong track record in peer-reviewed conferences/journals on planning and scheduling
and would like to apply to your position that I found in the mailing list.

I am a quick, clean programmer who can write efficient, maintainable and readable programs,
favoring collaboration, code review, good coding standards and other miscellaneous programmer's virtue,
as evidenced in my open source activities on github:
1) The first prototypes of the most complicated systems/projects are finished within 2 weeks.
2) Most projects are tested under Continuous Integration services for reliability.
 % I occasionally accept bug fixes, merged without breaking the code.
3) I regularly write concise documentation, using well-defined terms and ideas.
% 4) With an ability to fully leverage the metaprogramming, I can write even the most
%  complicated kinds of compile-time optimization within 500 lines of code.
% I can use these abilities for the rapid prototyping and the fast development cycles which ensure
%  the success of the entire project.
4) I have projects of more than 50 stars which occasionally gets some requests and questions from the users.
 I help my users understanding and using my product, through which I improve the documentation or the setup process.

I contributed to several projects of a former masters graduate student
by developing the RESTful API for a solver as well as running the
experiments for his algorithm. The former was used in his graduate thesis,
and the latter became a paper in HSDIP workshop in 2016.

I received an award from Japanese Society for Artificial Intelligence,
the largest local AI conference in Japan.
Although not being competitive/peer reviewed in nature,
this is a place for networking where most Japanese AI researchers attend,
and the award is a proof of excellence among Japanese researchers.

I worked on planning under uncertainty during the internship at IBM Research Ireland.
My latest publication in IJCAI is based on a hybrid system which uses
AO* for the low-level journey planning and A* for higher-level task scheduling
which invokes AO* multiple times. 
During the development, I significantly modified the AO* code, which was originally designed for one-off planning,
so that it integrates well to the higher-level planning code with additional caching between multiple invocations.
I also have familiarity with LAO* and more popular algorithms such as LRTA* or DP for MDP planning.

I also contributed to your SIGAPS project with a large-scale
manufacturing domain. My paper on ICAPS14 and ICAPS15 handles such
problems effectively through an automated Knowledge Engineering
techniques (or macro actions).

I received a government-led research funding which offers
stipends and independent research budget (1,000,000JPY annual) and will continue to collect external fundings.
The application goes through a standard process where the applicant writes 
a proposal and the process is highly selective (\href{https://www.jsps.go.jp/j-pd/pd_saiyo.html}{21.8\%}).

% Moreover, I am a fast learner. I wrote my first conference paper within 6 months after changing the
% major from Traffic Simulation to Heuristic Search (which I have no
% experience at that time), moving in to the different laboratory as a stranger.

% I am a fast and accurate writer, with a keen eye for detail and I should
% be very grateful for the opportunity to progress to market reporting. I
% am able to take on the responsibility of this position immediately, and
% have the enthusiasm and determination to ensure that I make a success
% of it.

% PARAGRAPH THREE: Request a personal interview and indicate your
 % flexibility as to the time and place. Repeat your phone number in the
 % letter. End the letter by thanking the employer for taking the time
 % to consider your credentials.\\ 

I worked on both the fundamental research and the application-oriented
research. I am excited about using my specialization for the real
world, fast pased development.
Thank you for the valuable time to consider this application. I look forward to hearing from you in the near future.

Sincerely yours,

\begin{flushright}
 Masataro Asai
\end{flushright}

\end{document}