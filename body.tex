% 
% 
% * Maintain 80 characters / line.
% * too much ``''s make the sentence look scattered and visually less recognizable. ``e.g.'' also.
% * \em, \bf, \it are all obsolete \TeX primitives, and it does not take effect properly ---
%   for example, {\bf {\it aaa}} shows ``aaa'' in italic but NOT IN BOLD.
%   Use \emph{}, \textit{}, \textbf{} and so on.
% 
% * Always use \ff, \fd, \cea, or other abbreviation commands in abbrev.sty.
%   Inconsistent notations would not be the major reason for rejection, but 
%   they eventually severely harm the impression of the paper.
% 
% * use of footnotes should be minimized.
% * IPC2011 should always be \ipc . The definition can later be modified in abbrev.sty .
% * prefer separated words over hyphened words. domain
%   independent>domain-independent, planner independent >
%   planner-independent.
% * Table, Figure, Fig., should not be used directly. Always use \refig and \reftbl. When the development flag is enabled, direct use of \ref signals an error.
% * Caption ends with a period.
% 
% 
\begin{abstract}
AAA!
\end{abstract}

\section{\texttt{aaai-template}}\label{aaai-template}

For the frequent attendants of top-tier AI conferences!
This repository includes templates and makefiles for:

\begin{itemize}

\item
  AAAI style
\item
  ECAI style
\item
  IJCAI style
\item
  JAIR style
\item
  NIPS style
\item
  ICML style
\item
  ICLR style
\item
  CVPR style
\end{itemize}

Requirements: GNU Make, TexLive, inkscape, perl (see dependency.sh)

\textbf{Update:} AAAI Press recently made a significant change to the
camera-ready requirements (such as
https://www.aaai.org/Publications/Author/icaps-submit.php). To address
the requirement we changed the file structure --- see below.

\section{File structure}

\begin{itemize}

\item
  \texttt{main.tex}

  \begin{itemize}
  
  \item
    toplevel file for the main paper, only containing the preamble.
  \end{itemize}
\item
  \texttt{body.tex}

  \begin{itemize}
  
  \item
    the text inside
    \texttt{\textbackslash{}begin\{document\}}--\texttt{\textbackslash{}end\{document\}}
    for \texttt{main.tex}.
  \end{itemize}
\item
  \texttt{supplemental.tex}

  \begin{itemize}
  
  \item
    toplevel text file for the supplementary material. Unlike the main
    paper, the text should be in this file too.
  \item
    \texttt{supplemental.tex} and \texttt{main.tex} can cross-reference
    the figures, tables and sections each other.
  \end{itemize}
\item
  \texttt{common-header.sty}

  \begin{itemize}
  
  \item
    Part of the preamble shared by \texttt{main.tex} and
    \texttt{supplemental.tex}.
  \item
    This file also contains the specific code for each conference. You
    should uncomment the code for the conference you plan to submit
  \end{itemize}
\item
  \texttt{commands-general.sty}

  \begin{itemize}
  
  \item
    general custom commands
  \end{itemize}
\item
  \texttt{commands-abbrev.sty}

  \begin{itemize}
  
  \item
    custom commands (only for the abbreviation)
  \end{itemize}
\end{itemize}

\section{Usage notes}

\begin{itemize}
\item
  It encourages the use of Inkskape to
  prepare svg images in \texttt{img} subdirectory, which are
  automatically compiled into pdf figures. (This is a recommended way,
  since pngs are bitmaps and doesnt print nicely.)

  On OSX, inkscape is available from \texttt{brew\ install\ inkscape}
\item
  \texttt{make\ submission}, \texttt{make\ archive},
  \texttt{make\ arxiv} : These \texttt{make} targets will create a
  \texttt{submission} directory and prepares the camera-ready tex files.
  There are sometimes extensive instruction for preparing the
  camera-ready submission, such as
  https://www.aaai.org/Publications/Author/icaps-submit.php .

  \begin{itemize}
  
  \item
    These camera-ready submissions do not allow the use of
    \texttt{\textbackslash{}input\{\}} command. When you run
    \texttt{make\ submission}, the results generated in the directory
    will have

    \begin{itemize}
    
    \item
      a single, flattened tex file whose \texttt{\textbackslash{}input}
      commands are inlined completely
    \item
      All image files referenced by the text are renamed and put in this
      root directory (AAAI Press does not allow putting images in the
      nested subdirectories)
    \item
      Garbage files (log files etc.) are removed.
    \end{itemize}
  \item
    \textbf{Usage note}: all \texttt{\textbackslash{}input\{\}} commands
    must be at the beginning of line, nothing before or after it.
    Otherwise it may remove some necessary text
  \item
    \texttt{make\ archive} compresses the \texttt{submission/} directory
    and create a zip or a tar.gz file. AAAI Press receive the zip file
    only, but this feature is also useful when submitting to Arxiv.

    \begin{itemize}
    
    \item
      Style files are removed (they are not allowed).
    \end{itemize}
  \item
    \texttt{make\ arxiv} is same as \texttt{make\ archive}, but does not
    remove the style files.
  \end{itemize}
\item
  \texttt{make\ auto} watches the source files and builds the pdf when
  they are updated. Requirements: \texttt{inotify-tools} package (it
  uses \texttt{inotifywait} for watching files and also sends messages
  via inotify notification popup window)
\item
  It also checks for the paper length limit and overful/underful hboxes,
  very useful in the last-limit optimization for overlength papers.
\item
  This template also supports JSAI, a local Japanese non-refereed
  confernece.
\end{itemize}



\section{If you have space in your paper, credit me}

Like this! \cite{aaai-template}
Or you can also use \texttt{nocite} command like this. \nocite{aaa-template}

\section{Introduction}

BBB! \refig{fig:ip} \cite{Asai2016}

\begin{figure}[tb]
 \includegraphics[width=\linewidth]{img/static/ip.png}
 \caption{This is Invasion Percolation}
 \label{fig:ip}
\end{figure}

\lmcut, \mands, \pdb, \ff, \ce, \cg, \ad, \lc heuristics.

In math mode,

\[
 \lmcut, \mands, \pdb, \ff, \ce, \cg, \ad, \lc.
\]

\lmcuto, \mandso, \ffo, \ceo, \cgo, \ado, \gco, \lco heuristics.

In math mode,

\[
 \lmcuto, \mandso, \ffo, \ceo, \cgo, \ado, \gco, \lco.
\]

% \begin{minted}{common-lisp}
% (defun factorial (n)
%   (if (zerop n)
%       1
%       (* n (factorial (1- n)))))
% \end{minted}

