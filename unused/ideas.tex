\todo{For Anna --- just a series of ideas, no real practice. The text is a sketch. Do not read this section.}

In \*a*, we first create a map of the plateau based on the path similarity
distance. This map can have an implicit or explicit representation.
Using this map, we diversify the order of expanding each node.

In RRT the map is never created explicitly.
In fact, when there is n nodes, creating such a map requires n(n-1) computation of the distance
between them and this could be expensive. Instead, RRT randomly selects a state
and find the nearest node in the tree. Nearest neighbor search is O(N), and
approximated NNS is O(log N).

\subsection{Distance Function}
\label{sec-2-1}

The selection of path similarity distance is very important.
In previous attempts to adopt RRT in classical planning, such as
RRT-Plan or RPT, they use Plan Distances between the leaf states as the
metric. This is not so useful due to many reasons:

\begin{itemize}
\item It requires a computationally expensive explicit search each time.
\item Computing the relaxed plan distance is still expensive.
\item Plan distance does not reflect how similar these states are.
\begin{itemize}
\item Example 1: Symmetry: Although symmetric states are naturally very
"similar", their plan distance may be very large, or even
infinite. -- Suppose we are standing at the center of a straight
road which is 2km long. We can go 1km to the left or 1km to the
right. There are two goals in both ends. Although the two goals and the
paths are symmetric, the goal distance is 2km long, which is a largest
diameter of this search space.
\item Example 2: Partial Order: Suppose we are in a 3x2 grid where we can go
up or right only. Initial/Goal State is (0,0)/(2,1). Suppose the search
frontiers have the paths (0,0)-(0,1)-(1-1) and (0,0)-(1,0)-(2-0).
They both have a parent node
(1,0) and the paths before this meeting point are the same in terms of
partial order. These are not symmetric. Although these frontiers both reach
the same goal state in 1 step, and their prefix are the same, their
plan distance is infinite since we cannot go down or left.
\end{itemize}
\end{itemize}

\begin{verbatim}
g----------s----------g
\end{verbatim}

\begin{verbatim}
+--o--g
|  |  |
s--+--o
\end{verbatim}

We have several possibilities for computing the similarity distance. Here
are some ideas:

\begin{enumerate}
\item We count the difference of the plan according to the D\(_{\text{stability}}\) in
Diversity Planning(Nguyen 12). However, we also want to take the following elements
into account:
\item the length of the shared prefix of the paths s0-s1 and s0-s2.
\begin{itemize}
\item This is fast but not informative, and exploits no information in
Symmetry and Partial Ordering.
\item This only provides the upper bound of the plan similarity because,
although the path s0-s1 is optimal in the admissible search, there
may be several such optimal paths, and this path prefix distance
does not take those multiple paths into consideration.
\end{itemize}
\item Convert the path into the partial order plan and compute the graph
similarity. I suppose Graph Similarity is computationally hard problem
but is in practice very fast. This would make the similarity
more accurate.
\item Path symmetry. This would also make the similarity more accurate.
\end{enumerate}

Note: these do not require those paths to be completely symmetric or
completely isomorphic partial order path. The benefit is that we can delay
the expansion of "mostly" symmetric path/isomorphic partial order path,
although we may not be able to prune them completely.

Note: The role is similar to globally admissible heuristics, which is in turn
found to be a "path-dependent heuristics". However, these information are
highly likely to be considered by an ideal "almost perfect" heuristics.
Those ideal heuristics take into consideration \emph{any possible} information
toward the goal, but \emph{not} about the already-known information.

Note: Another problem in previous RRT-based planners is how to query a reachable
state randomly, uniformly and efficiently. They run "generate, test and
dispose" approach, however we do not need this because we only apply our
ordering algorithm to the search frontier with best-f. Open-nodes are
always reached from the initial states.

Now the problem is how to diversify the selection from the best-f states,
without creating a distance table which runs n(n-1)/2 iteration.
We instead create the table in an incremental fashion during the search.
We expect a new f-value is obtained before every node is expanded.
Since it does not modify the expansion order regarding f, the algorithm
does not harm the admissibility.

Consider we have a set of nodes n1-n5. It has the following
pairwise distances, which is actually unknown prior to the search.

\begin{center}
\begin{tabular}{r|rrrrr|}
 &  &  &  &  & \\
 & 1 & 2 & 3 & 4 & 5\\
\hline
1 & $\backslash$ & 1 & 7 & 2 & 9\\
2 & 1 & $\backslash$ & 3 & 8 & 5\\
3 & 7 & 3 & $\backslash$ & 4 & 6\\
4 & 2 & 8 & 4 & $\backslash$ & 0\\
5 & 9 & 5 & 6 & 0 & $\backslash$\\
\hline
\end{tabular}
\end{center}

The first node is selected arbitrarily. Let's assume it is n1.
The second node is the farthest node from the root node, which is n5. It
takes 4 distance computation.
Now which node should be the third?
We want to select a node which is diverse from both nodes.
From the table, we estimate n3 is far from both node.
When multiple nodes are accounted, we use a harmonic mean of the distances.
It implies that if some state returns a distance 0 (e.g. they are
symmetric), it is considered only in the last. (of course, it is possible to
prune these states\ldots{})
\subsection{talk with alexf}
\label{sec-2-2}

is path distance really needed? (maybe just a state similarity is enough)
compute plan similarity with zobrist-hash like fast method?
