% \subsubparagraph{\sota heuristics}

% These intractable heuristics are of course very hard and expensive to
% compute. However, even the practical and tractable \sota heuristic
% functions for STRIPS Planning, such as \lmcut\cite{Helmert2009} and
% IP-based heuristics[???], are so heavily CPU intensive that they
% outweighs the space-greedy nature of \astar. Also, compared to the other
% functions, these functions dominate the search time over the other
% factors of planning algorithms, such as node insertion and deletion.

% In contrast, another class of \sota heuristic functions called abstraction-based heuristics
% such as PDB \cite{edelkamp2001planning} and M\&S \cite{helmert2007flexible}
% are known to allow for faster node expansion while providing a good estimate.
% In particular, depending on the domain and Merge \& Shrinke strategy,
% M\&S sometimes yields a perfect heuristics.
% However, they tend to consume large amount of preprocessing time and also
% the large amount of memory to constract and maintain the abstraction
% space,
% and the performance tends to be memory-bounded rather than CPU-bounded,
% which is also doubled by the fast expansion.
