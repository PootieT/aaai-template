\documentclass[10pt,letterpaper]{article}
\usepackage{aaai}
\usepackage{xparse}
\usepackage{times}
\usepackage{helvet}
\usepackage{courier}

\usepackage{amsmath}
\usepackage{amssymb}
\usepackage{xspace}
\usepackage{relsize}
\usepackage{aaai_my}
\usepackage{graphicx}
\usepackage{abbrev}
\usepackage{multirow}
\usepackage{xr}
\externaldocument{asai}
\renewcommand{\thetable}{S\arabic{table}}
\renewcommand{\thefigure}{S\arabic{figure}}

\setlength{\pdfpagewidth}{8.5in}
\setlength{\pdfpageheight}{11in}
\pagestyle{empty}
\frenchspacing
% \setlength{\floatsep}{1mm}
% \setlength{\textfloatsep}{1mm}
% \setlength{\abovecaptionskip}{1mm}
% \setlength{\belowcaptionskip}{1mm}
% \setlength{\abovedisplayskip}{1mm}
% \setlength{\belowdisplayskip}{1mm}
% \setlength{\arraycolsep}{0.5mm}
\setlength{\tabcolsep}{0.1em}

\nocopyright
\author{Submision 974}
\title{Tiebreaking Strategies for Classical Planning Using $A^*$ Search\\ (Supplemental Material)}
\begin{document}
\maketitle

\section{\refig{plateau-noh-full}: Colored, fully labeled version of \refig{fig:plateau-noh}}

\begin{figure}[hb]
 \includegraphics{tables/aaai16-frontier/aaai16prelim3/lmcut_frontier_noh-front.pdf}
 \caption{
 The number of nodes with $f=f^*$ (y-axis) compared to the
 total number of nodes in the search space (x-axis) with $f\leq f^*$
 on 1104 IPC benchmark problems,
 using a modified Fast Downward (with \lmcut) which 
 generates all nodes $f\leq f^*$.
 Blue lines are $y=x \times 10^n$ for $n=\pm1,\pm2,\cdots$.
 Many instances belongs to the region $y>x/10$.
 }
 \label{plateau-noh-full}
\end{figure}

\newpage

\section{\refig{plateau-full}: Colored, fully labeled version of \refig{plateau}}


\begin{figure}[htb]
 \includegraphics{tables/aaai16-frontier/aaai16prelim3/lmcut_frontier-front.pdf}
 \caption{
 Similar to \refig{plateau-noh-full}; $y$-axis shows
 number of nodes with $f=f^*, h=0$, which forms the final
  plateau when $h$-based tie-breaking is enabled.
  Note that many \pddl{Openstacks} and \pddl{Cybersec} instances are near the $y=x$ line.
  %Due to space, we do not show all labels for point types. 
 }
 \label{plateau-full}
\end{figure}

\newpage
\section{\refig{plateau-zerocost-full}: Colored, fully labeled version of \refig{plateau-zerocost}}

\begin{figure}[htb]
 \includegraphics{tables/aaai16-frontier/zerocost/lmcut_frontier-front.pdf}
 \caption{Similar to \refig{plateau}, but for 620 instances from our 
 \emph{zerocost domains} (\refsec{sec:zerocost-domains}),
 where zero-cost actions induce very large plateaus.
 Even with $h$ tie-breaking, these domains force the planner
 to search much larger plateau.
 }
 \label{plateau-zerocost-full}
\end{figure}

\newpage
\section{\refig{f-h-eval-full}: Colored, fully labeled version of \refig{f-h-eval}}

\begin{figure}[htb]
 \centering \relsize{-3}
 \includegraphics{tables/aaai16-30min-5min-cut/aaai16prelim3/evaluated-lmcut_ff-lmcut_lf.pdf}
 \caption{Comparisons of the number evaluations between simple \lifo and
 \fifo second-level tie-breaking, with first-level $h$
 tie-breaking. \lifo evaluates less than $1/10$ of the nodes evaluated
 by \fifo in Cybersec and Openstacks.}
 \label{f-h-eval-full}
\end{figure}

\clearpage
\onecolumn
\section{Full Tables for \reftbl{depth} : Part 1, IPC domains, \lmcut}

\begin{table*}[htb]
 {
 \centering
 \begin{tabular}{|c|c|c|c|c|c|c|c|c|c||c|c|c|}
\hline
 & \multicolumn{4}{|c|}{Coverages}
 & \multicolumn{5}{|c||}{Coverages (mean$\pm$sd)}
 & \multicolumn{3}{|c|}{Wilcoxon $p$ vs $[h,\rd,\ro]$} \\
\hline                                    
 Domain &  $[h,\fifo]$ &  $[h,\lifo]$ &  $[\fifo]$ &  $[\lifo]$ &  $[h,\fd,\ro]$ &  $[h,\ld,\ro]$ &  $[h,\rd,\ro]$ &  $[\rd,\ro]$ &  $[h,\ro]$ & $[h,\fd,\ro]$   & $[h,\ld,\ro]$   & $[h,\ro]$    \\
\hline                                    
 IPC Benchmark(1104) &  558 &  565 &  442 &  556 &  554.6\spm{}0.8 &  568.3\spm{}1.8 &  \textbf{570.6\spm{}1.5} &  560.0\spm{}0.9 &  559.8\spm{}1.0 &  \textbf{0.0} &  \textbf{.01} &  \textbf{0.0}  \\
\hline                                    
    {\relsize{-1}airport(50)} &  \textbf{27} &  26 &  18 &  26 &  25.6\spm{}0.5 &  25.8\spm{}0.6 &  25.9\spm{}0.5 &  21.0\spm{}0.0 &  26.0\spm{}0.0 &  .26 &  .72 &  .58  \\
%   {\relsize{-1}barman-opt11(20)} &  0 &  0 &  0 &  0 &  0.0\spm{}0.0 &  0.0\spm{}0.0 &  0.0\spm{}0.0 &  0.0\spm{}0.0 &  0.0\spm{}0.0 &  1.0 &  1.0 &  1.0  \\
%   {\relsize{-1}blocks(35)} &  28 &  28 &  26 &  26 &  28.0\spm{}0.0 &  28.0\spm{}0.0 &  28.0\spm{}0.0 &  27.0\spm{}0.0 &  28.0\spm{}0.0 &  1.0 &  1.0 &  1.0  \\
    {\relsize{-1}cybersec(19)} &  2 &  3 &  0 &  3 &  2.0\spm{}0.0 &  7.3\spm{}1.5 &  \textbf{9.6\spm{}1.1} &  7.8\spm{}0.7 &  4.4\spm{}1.0 &  \textbf{0.0} &  \textbf{.01} &  \textbf{0.0}  \\
%   {\relsize{-1}depot(22)} &  6 &  6 &  5 &  5 &  6.0\spm{}0.0 &  6.0\spm{}0.0 &  6.0\spm{}0.0 &  6.0\spm{}0.0 &  6.0\spm{}0.0 &  1.0 &  1.0 &  1.0  \\
%   {\relsize{-1}driverlog(20)} &  13 &  13 &  12 &  13 &  13.0\spm{}0.0 &  13.0\spm{}0.0 &  13.0\spm{}0.0 &  13.0\spm{}0.0 &  13.0\spm{}0.0 &  1.0 &  1.0 &  1.0  \\
%   {\relsize{-1}elevators-opt11(20)} &  15 &  15 &  14 &  15 &  15.0\spm{}0.0 &  15.0\spm{}0.0 &  15.0\spm{}0.0 &  14.8\spm{}0.4 &  15.0\spm{}0.0 &  1.0 &  1.0 &  1.0  \\
%   {\relsize{-1}floortile-opt11(20)} &  6 &  6 &  6 &  6 &  6.0\spm{}0.0 &  6.0\spm{}0.0 &  6.0\spm{}0.0 &  6.0\spm{}0.0 &  6.0\spm{}0.0 &  1.0 &  1.0 &  1.0  \\
%   {\relsize{-1}freecell(80)} &  9 &  9 &  8 &  9 &  9.0\spm{}0.0 &  9.0\spm{}0.0 &  9.0\spm{}0.0 &  9.0\spm{}0.0 &  9.0\spm{}0.0 &  1.0 &  1.0 &  1.0  \\
%   {\relsize{-1}grid(5)} &  1 &  1 &  1 &  1 &  1.0\spm{}0.0 &  1.0\spm{}0.0 &  1.0\spm{}0.0 &  1.0\spm{}0.0 &  1.0\spm{}0.0 &  1.0 &  1.0 &  1.0  \\
%   {\relsize{-1}gripper(20)} &  6 &  6 &  6 &  6 &  6.0\spm{}0.0 &  6.0\spm{}0.0 &  6.0\spm{}0.0 &  6.0\spm{}0.0 &  6.0\spm{}0.0 &  1.0 &  1.0 &  1.0  \\
%   {\relsize{-1}hanoi(30)} &  12 &  12 &  12 &  12 &  12.0\spm{}0.0 &  12.0\spm{}0.0 &  12.0\spm{}0.0 &  12.0\spm{}0.0 &  12.0\spm{}0.0 &  1.0 &  1.0 &  1.0  \\
    {\relsize{-1}logistics00(28)} &  \textbf{20} &  \textbf{20} &  16 &  18 &  \textbf{20.0\spm{}0.0} &  \textbf{20.0\spm{}0.0} & \textbf{20.0\spm{}0.0} &  \textbf{20.0\spm{}0.0} &  \textbf{20.0\spm{}0.0} &  1.0 &  1.0 &  1.0  \\
    {\relsize{-1}miconic(150)} &  \textbf{140} & \textbf{140} &  68 &  \textbf{140} &  \textbf{140.0\spm{}0.0} & \textbf{140.0\spm{}0.0} & \textbf{140.0\spm{}0.0} &  135.6\spm{}0.5 &  \textbf{140.0\spm{}0.0} &  1.0 &  1.0 &  1.0  \\
    {\relsize{-1}mprime(35)} &  21 &  21 &  19 &  \textbf{22} &  20.9\spm{}0.3 &  20.9\spm{}0.3 &  20.9\spm{}0.3 &  21.0\spm{}0.0 &  20.9\spm{}0.3 &  1.0 &  1.0 &  1.0  \\
%   {\relsize{-1}mystery(30)} &  15 &  16 &  15 &  15 &  15.0\spm{}0.0 &  15.0\spm{}0.0 &  15.0\spm{}0.0 &  15.8\spm{}0.4 &  15.0\spm{}0.0 &  1.0 &  1.0 &  1.0  \\
    {\relsize{-1}nomystery-opt11(20)} &  \textbf{14} &  \textbf{14} &  12 &  13 &  \textbf{14.0\spm{}0.0} &  \textbf{14.0\spm{}0.0} &  \textbf{14.0\spm{}0.0} &  13.8\spm{}0.4 &  \textbf{14.0\spm{}0.0} &  1.0 &  1.0 &  1.0  \\
    {\relsize{-1}openstacks-opt11(20)} &  11 & \textbf{18} &  11 & \textbf{18} &  10.0\spm{}0.0 &  \textbf{18.0\spm{}0.0} &  \textbf{18.0\spm{}0.0} & \textbf{18.0\spm{}0.0} &  11.6\spm{}0.5 &  \textbf{0.0} &  1.0 &  \textbf{0.0}  \\
%   {\relsize{-1}parcprinter-opt11(20)} &  13 &  13 &  12 &  13 &  13.0\spm{}0.0 &  13.0\spm{}0.0 &  13.0\spm{}0.0 &  13.0\spm{}0.0 &  13.0\spm{}0.0 &  1.0 &  1.0 &  1.0  \\
%   {\relsize{-1}parking-opt11(20)} &  1 &  1 &  1 &  1 &  1.0\spm{}0.0 &  1.0\spm{}0.0 &  1.0\spm{}0.0 &  1.0\spm{}0.0 &  1.0\spm{}0.0 &  1.0 &  1.0 &  1.0  \\
%   {\relsize{-1}pathways(30)} &  5 &  5 &  4 &  5 &  5.0\spm{}0.0 &  5.0\spm{}0.0 &  5.0\spm{}0.0 &  5.0\spm{}0.0 &  5.0\spm{}0.0 &  1.0 &  1.0 &  1.0  \\
%   {\relsize{-1}pegsol-opt11(20)} &  17 &  17 &  17 &  17 &  17.0\spm{}0.0 &  17.0\spm{}0.0 &  17.0\spm{}0.0 &  17.0\spm{}0.0 &  17.0\spm{}0.0 &  1.0 &  1.0 &  1.0  \\
    {\relsize{-1}pipesworld-notankage(50)} &  \textbf{15} &  14 &  13 &  13 &  14.1\spm{}0.3 &  14.3\spm{}0.5 &  14.2\spm{}0.4 &  14.2\spm{}0.4 &  14.9\spm{}0.3 &  .58 &  .65 &  \textbf{0.0}  \\
%   {\relsize{-1}pipesworld-tankage(50)} &  8 &  8 &  7 &  8 &  8.0\spm{}0.0 &  8.0\spm{}0.0 &  8.0\spm{}0.0 &  8.0\spm{}0.0 &  8.0\spm{}0.0 &  1.0 &  1.0 &  1.0  \\
%   {\relsize{-1}psr-small(50)} &  48 &  48 &  48 &  48 &  48.0\spm{}0.0 &  48.0\spm{}0.0 &  48.0\spm{}0.0 &  48.0\spm{}0.0 &  48.0\spm{}0.0 &  1.0 &  1.0 &  1.0  \\
%   {\relsize{-1}rovers(40)} &  7 &  7 &  7 &  7 &  7.0\spm{}0.0 &  7.0\spm{}0.0 &  7.0\spm{}0.0 &  7.0\spm{}0.0 &  7.0\spm{}0.0 &  1.0 &  1.0 &  1.0  \\
    {\relsize{-1}scanalyzer-opt11(20)} &  \textbf{10} &  \textbf{10} &  4 &  \textbf{10} &  \textbf{10.0\spm{}0.0} &  \textbf{10.0\spm{}0.0} &  \textbf{10.0\spm{}0.0} &  9.0\spm{}0.0 & \textbf{10.0\spm{}0.0} &  1.0 &  1.0 &  1.0  \\
%   {\relsize{-1}sokoban-opt11(20)} &  19 &  19 &  19 &  19 &  19.0\spm{}0.0 &  19.0\spm{}0.0 &  19.0\spm{}0.0 &  19.0\spm{}0.0 &  19.0\spm{}0.0 &  1.0 &  1.0 &  1.0  \\
%   {\relsize{-1}storage(30)} &  14 &  14 &  14 &  14 &  14.0\spm{}0.0 &  14.0\spm{}0.0 &  14.0\spm{}0.0 &  14.4\spm{}0.5 &  14.0\spm{}0.0 &  1.0 &  1.0 &  1.0  \\
%   {\relsize{-1}tidybot-opt11(20)} &  12 &  12 &  11 &  11 &  12.0\spm{}0.0 &  12.0\spm{}0.0 &  12.0\spm{}0.0 &  11.8\spm{}0.4 &  12.0\spm{}0.0 &  1.0 &  1.0 &  1.0  \\
%   {\relsize{-1}tpp(30)} &  6 &  6 &  6 &  6 &  6.0\spm{}0.0 &  6.0\spm{}0.0 &  6.0\spm{}0.0 &  6.0\spm{}0.0 &  6.0\spm{}0.0 &  1.0 &  1.0 &  1.0  \\
%   {\relsize{-1}transport-opt11(20)} &  6 &  6 &  6 &  6 &  6.0\spm{}0.0 &  6.0\spm{}0.0 &  6.0\spm{}0.0 &  6.0\spm{}0.0 &  6.0\spm{}0.0 &  1.0 &  1.0 &  1.0  \\
%   {\relsize{-1}visitall-opt11(20)} &  10 &  10 &  9 &  10 &  10.0\spm{}0.0 &  10.0\spm{}0.0 &  10.0\spm{}0.0 &  10.0\spm{}0.0 &  10.0\spm{}0.0 &  1.0 &  1.0 &  1.0  \\
    {\relsize{-1}woodworking-opt11(20)} &  10 &  10 &  6 &  9 &  10.0\spm{}0.0 &  10.0\spm{}0.0 &  10.0\spm{}0.0 &  \textbf{11.8\spm{}0.4} &  10.0\spm{}0.0 &  1.0 &  1.0 &  1.0  \\
    {\relsize{-1}zenotravel(20)} &  \textbf{11} &  \textbf{11} &  9 &  \textbf{11} &  \textbf{11.0\spm{}0.0} &  \textbf{11.0\spm{}0.0} &  \textbf{11.0\spm{}0.0} & \textbf{11.0\spm{}0.0} & \textbf{11.0\spm{}0.0} &  1.0 &  1.0 &  1.0 \\\hline
    Zerocost(620) &  256 &  279 &  212 &  281 &  249.1\spm{}1.8 &  280.2\spm{}7.9 &  \textbf{287.2\spm{}2.4} &  280.2\spm{}4.2 &  264.9\spm{}1.8 &  \textbf{0.0} &  \textbf{.02} &  \textbf{0.0}  \\
  \hline                                    
    {\relsize{-1}airport-fuel(20)} &  \textbf{15} &  13 &  7 &  \textbf{15} &  14.2\spm{}0.9 &  13.8\spm{}0.6 &  14.4\spm{}0.7 &  10.4\spm{}0.5 &  14.4\spm{}0.7 &  .49 &  .06 &  1.0  \\
    {\relsize{-1}blocks-stack(20)} &  \textbf{17} &  \textbf{17} &  15 &  \textbf{17} &  \textbf{17.0\spm{}0.0} &  \textbf{17.1\spm{}0.3} &  \textbf{17.0\spm{}0.0} &  16.0\spm{}0.0 &  \textbf{17.0\spm{}0.0} &  1.0 &  .37 &  1.0  \\
%   {\relsize{-1}depot-fuel(22)} &  6 &  6 &  4 &  6 &  6.0\spm{}0.0 &  6.0\spm{}0.0 &  6.0\spm{}0.0 &  6.0\spm{}0.0 &  6.0\spm{}0.0 &  1.0 &  1.0 &  1.0  \\
%   {\relsize{-1}driverlog-fuel(20)} &  8 &  8 &  7 &  8 &  8.0\spm{}0.0 &  7.2\spm{}0.7 &  8.0\spm{}0.0 &  8.0\spm{}0.0 &  8.0\spm{}0.0 &  1.0 &  \textbf{.01} &  1.0  \\
    {\relsize{-1}elevators-up(20)} &  7 &  \textbf{13} &  7 &  \textbf{13} &  5.3\spm{}0.5 &  8.8\spm{}0.9 &  9.4\spm{}1.1 &  8.2\spm{}0.7 &  7.3\spm{}0.5 &  \textbf{0.0} &  .25 &  \textbf{0.0}  \\
%   {\relsize{-1}floortile-ink(20)} &  8 &  8 &  8 &  8 &  8.0\spm{}0.0 &  8.0\spm{}0.0 &  8.1\spm{}0.3 &  8.0\spm{}0.0 &  8.3\spm{}0.5 &  .37 &  .37 &  0.3  \\
    {\relsize{-1}freecell-move(20)} &  4 &  19 &  4 &  19 &  4.0\spm{}0.0 &  \textbf{19.4\spm{}0.5} &  16.5\spm{}0.7 &  16.6\spm{}0.8 &  5.0\spm{}0.4 &  \textbf{0.0} &  \textbf{0.0} &  \textbf{0.0}  \\
%   {\relsize{-1}grid-fuel(5)} &  1 &  1 &  1 &  1 &  1.0\spm{}0.0 &  1.0\spm{}0.0 &  1.0\spm{}0.0 &  1.0\spm{}0.0 &  1.0\spm{}0.0 &  1.0 &  1.0 &  1.0  \\
%   {\relsize{-1}gripper-move(20)} &  7 &  7 &  7 &  7 &  6.0\spm{}0.0 &  6.0\spm{}0.0 &  6.0\spm{}0.0 &  7.0\spm{}0.0 &  7.0\spm{}0.0 &  1.0 &  1.0 &  \textbf{0.0}  \\
%   {\relsize{-1}hiking-fuel(20)} &  9 &  9 &  8 &  9 &  9.0\spm{}0.0 &  9.0\spm{}0.0 &  9.0\spm{}0.0 &  9.0\spm{}0.0 &  9.0\spm{}0.0 &  1.0 &  1.0 &  1.0  \\
%   {\relsize{-1}logistics00-fuel(28)} &  16 &  16 &  15 &  16 &  15.0\spm{}0.0 &  15.0\spm{}0.0 &  15.0\spm{}0.0 &  16.0\spm{}0.0 &  16.0\spm{}0.0 &  1.0 &  1.0 &  \textbf{0.0}  \\
    {\relsize{-1}miconic-up(30)} &  16 &  17 &  10 &  17 &  15.4\spm{}0.5 &  18.0\spm{}1.3 &  19.8\spm{}1.0 &  \textbf{20.4\spm{}1.0} &  17.0\spm{}0.4 &  \textbf{0.0} &  \textbf{0.0} &  \textbf{0.0}  \\
    {\relsize{-1}mprime-succumb(35)} &  15 &  14 &  12 &  14 &  15.8\spm{}0.7 &  18.7\spm{}3.9 &  \textbf{20.1\spm{}0.7} &  18.6\spm{}2.0 &  17.9\spm{}0.5 &  \textbf{0.0} &  .23 &  \textbf{0.0}  \\
%   {\relsize{-1}mystery-feast(20)} &  7 &  5 &  5 &  5 &  7.2\spm{}0.4 &  6.2\spm{}0.7 &  7.2\spm{}0.4 &  7.2\spm{}0.7 &  7.3\spm{}0.5 &  1.0 &  \textbf{0.0} &  .65  \\
%   {\relsize{-1}nomystery-fuel(20)} &  10 &  10 &  9 &  10 &  10.0\spm{}0.0 &  10.0\spm{}0.0 &  10.0\spm{}0.0 &  9.4\spm{}0.5 &  10.0\spm{}0.0 &  1.0 &  1.0 &  1.0  \\
%   {\relsize{-1}parking-movecc(20)} &  0 &  0 &  0 &  0 &  0.0\spm{}0.0 &  0.0\spm{}0.0 &  0.0\spm{}0.0 &  0.0\spm{}0.0 &  0.0\spm{}0.0 &  1.0 &  1.0 &  1.0  \\
%   {\relsize{-1}pathways-fuel(30)} &  5 &  5 &  4 &  5 &  4.0\spm{}0.0 &  4.2\spm{}0.4 &  4.4\spm{}0.5 &  4.8\spm{}0.4 &  4.4\spm{}0.5 &  \textbf{.03} &  .37 &  1.0  \\
    {\relsize{-1}pipesnt-pushstart(20)} &  8 &  8 &  6 &  7 &  8.0\spm{}0.0 &  8.6\spm{}1.3 &  \textbf{9.8\spm{}0.4} &  \textbf{9.8\spm{}0.4} &  8.5\spm{}0.5 &  \textbf{0.0} &  \textbf{.04} &  \textbf{0.0}  \\
    {\relsize{-1}pipesworld-pushend(20)} &  3 &  4 &  2 &  4 &  3.0\spm{}0.0 &  4.2\spm{}1.0 &  4.5\spm{}0.8 &  \textbf{5.4\spm{}0.8} &  3.9\spm{}0.3 &  \textbf{0.0} &  0.5 &  \textbf{.05}  \\
%   {\relsize{-1}psr-small-open(20)} &  19 &  19 &  19 &  19 &  18.0\spm{}0.0 &  19.0\spm{}0.0 &  19.0\spm{}0.0 &  19.0\spm{}0.0 &  19.0\spm{}0.0 &  \textbf{0.0} &  1.0 &  1.0  \\
%   {\relsize{-1}rovers-fuel(40)} &  8 &  8 &  7 &  9 &  8.0\spm{}0.0 &  8.0\spm{}0.0 &  8.0\spm{}0.0 &  9.0\spm{}0.0 &  8.0\spm{}0.0 &  1.0 &  1.0 &  1.0  \\
    {\relsize{-1}scanalyzer-analyze(20)} &  9 &  9 &  3 &  9 &  \textbf{9.7\spm{}0.6} &  9.3\spm{}0.5 &  9.1\spm{}0.3 &  7.4\spm{}1.0 &  9.1\spm{}0.3 &  \textbf{.02} &  0.3 &  1.0  \\
%   {\relsize{-1}sokoban-pushgoal(20)} &  18 &  18 &  18 &  18 &  18.0\spm{}0.0 &  17.0\spm{}0.0 &  17.9\spm{}0.3 &  17.0\spm{}0.0 &  18.0\spm{}0.0 &  .37 &  \textbf{0.0} &  .37  \\
%   {\relsize{-1}storage-lift(20)} &  4 &  4 &  4 &  4 &  4.0\spm{}0.0 &  5.0\spm{}1.2 &  4.4\spm{}0.5 &  4.6\spm{}0.5 &  4.6\spm{}0.5 &  \textbf{.03} &  .26 &  .41  \\
%   {\relsize{-1}tidybot-motion(20)} &  16 &  16 &  14 &  16 &  16.0\spm{}0.0 &  16.0\spm{}0.0 &  16.0\spm{}0.0 &  15.6\spm{}0.5 &  16.0\spm{}0.0 &  1.0 &  1.0 &  1.0  \\
    {\relsize{-1}tpp-fuel(30)} &  8 &  \textbf{11} &  7 &  \textbf{11} &  7.0\spm{}0.0 &  \textbf{11.0\spm{}0.0} &  \textbf{11.0\spm{}0.0} &  \textbf{11.0\spm{}0.0} &  8.1\spm{}0.3 &  \textbf{0.0} &  1.0 &  \textbf{0.0}  \\
    {\relsize{-1}woodworking-cut(20)} &  5 &  7 &  2 &  7 &  4.5\spm{}0.5 &  6.7\spm{}0.5 &  \textbf{8.6\spm{}0.9} &  7.8\spm{}0.7 &  7.1\spm{}0.3 &  \textbf{0.0} &  \textbf{0.0} &  \textbf{0.0}  \\
%   {\relsize{-1}zenotravel-fuel(20)} &  7 &  7 &  7 &  7 &  7.0\spm{}0.0 &  7.0\spm{}0.0 &  7.0\spm{}0.0 &  7.0\spm{}0.0 &  7.0\spm{}0.0 &  1.0 &  1.0 &  1.0 \\
\hline
 Total(1724) &  814 &  844 &  654 &  837 &  803.7\spm{}2.2 &  848.5\spm{}8.9 &  \textbf{857.8\spm{}2.9} &  840.2\spm{}4.4 &  824.7\spm{}2.1 &  \textbf{0.0} &  \textbf{.01} &  \textbf{0.0} \\\hline
\end{tabular}

 \caption{
 Full version of the upper half of \reftbl{depth} showing 
 the experiments on the IPC benchmark instances using \lmcut heuritics.
 Each cell shows the coverage of the domain solved with 5 min, 2GB.
 As in the original \reftbl{depth}, we highlighted the best results in
 \textbf{boldface} only when the maximum pairwise coverage difference $\mit{MaxDiff}>2$.
 }
 \label{lmcut-ipc-full}
 }
\end{table*}

\newpage
\section{Full Tables for \reftbl{depth} : Part 2, Zerocost domains, \lmcut}

\begin{table*}[htb]
 {
 \centering
 \begin{tabular}{|c|c|c|c|c|c|c|c|c|c||c|c|c|}
\hline
 & \multicolumn{4}{|c|}{Coverages}
 & \multicolumn{5}{|c||}{Coverages (mean$\pm$sd)}
 & \multicolumn{3}{|c|}{Wilcoxon $p$ vs $[h,\rd,\ro]$} \\
\hline                                    
 Domain &  $[h,\fifo]$ &  $[h,\lifo]$ &  $[\fifo]$ &  $[\lifo]$ &  $[h,\fd,\ro]$ &  $[h,\ld,\ro]$ &  $[h,\rd,\ro]$ &  $[\rd,\ro]$ &  $[h,\ro]$ & $[h,\fd,\ro]$   & $[h,\ld,\ro]$   & $[h,\ro]$    \\
\hline                                    
 IPC Benchmark(1104) &  558 &  565 &  442 &  556 &  554.6\spm{}0.8 &  568.3\spm{}1.8 &  \textbf{570.6\spm{}1.5} &  560.0\spm{}0.9 &  559.8\spm{}1.0 &  \textbf{0.0} &  \textbf{.01} &  \textbf{0.0}  \\
\hline                                    
    {\relsize{-1}airport(50)} &  \textbf{27} &  26 &  18 &  26 &  25.6\spm{}0.5 &  25.8\spm{}0.6 &  25.9\spm{}0.5 &  21.0\spm{}0.0 &  26.0\spm{}0.0 &  .26 &  .72 &  .58  \\
%   {\relsize{-1}barman-opt11(20)} &  0 &  0 &  0 &  0 &  0.0\spm{}0.0 &  0.0\spm{}0.0 &  0.0\spm{}0.0 &  0.0\spm{}0.0 &  0.0\spm{}0.0 &  1.0 &  1.0 &  1.0  \\
%   {\relsize{-1}blocks(35)} &  28 &  28 &  26 &  26 &  28.0\spm{}0.0 &  28.0\spm{}0.0 &  28.0\spm{}0.0 &  27.0\spm{}0.0 &  28.0\spm{}0.0 &  1.0 &  1.0 &  1.0  \\
    {\relsize{-1}cybersec(19)} &  2 &  3 &  0 &  3 &  2.0\spm{}0.0 &  7.3\spm{}1.5 &  \textbf{9.6\spm{}1.1} &  7.8\spm{}0.7 &  4.4\spm{}1.0 &  \textbf{0.0} &  \textbf{.01} &  \textbf{0.0}  \\
%   {\relsize{-1}depot(22)} &  6 &  6 &  5 &  5 &  6.0\spm{}0.0 &  6.0\spm{}0.0 &  6.0\spm{}0.0 &  6.0\spm{}0.0 &  6.0\spm{}0.0 &  1.0 &  1.0 &  1.0  \\
%   {\relsize{-1}driverlog(20)} &  13 &  13 &  12 &  13 &  13.0\spm{}0.0 &  13.0\spm{}0.0 &  13.0\spm{}0.0 &  13.0\spm{}0.0 &  13.0\spm{}0.0 &  1.0 &  1.0 &  1.0  \\
%   {\relsize{-1}elevators-opt11(20)} &  15 &  15 &  14 &  15 &  15.0\spm{}0.0 &  15.0\spm{}0.0 &  15.0\spm{}0.0 &  14.8\spm{}0.4 &  15.0\spm{}0.0 &  1.0 &  1.0 &  1.0  \\
%   {\relsize{-1}floortile-opt11(20)} &  6 &  6 &  6 &  6 &  6.0\spm{}0.0 &  6.0\spm{}0.0 &  6.0\spm{}0.0 &  6.0\spm{}0.0 &  6.0\spm{}0.0 &  1.0 &  1.0 &  1.0  \\
%   {\relsize{-1}freecell(80)} &  9 &  9 &  8 &  9 &  9.0\spm{}0.0 &  9.0\spm{}0.0 &  9.0\spm{}0.0 &  9.0\spm{}0.0 &  9.0\spm{}0.0 &  1.0 &  1.0 &  1.0  \\
%   {\relsize{-1}grid(5)} &  1 &  1 &  1 &  1 &  1.0\spm{}0.0 &  1.0\spm{}0.0 &  1.0\spm{}0.0 &  1.0\spm{}0.0 &  1.0\spm{}0.0 &  1.0 &  1.0 &  1.0  \\
%   {\relsize{-1}gripper(20)} &  6 &  6 &  6 &  6 &  6.0\spm{}0.0 &  6.0\spm{}0.0 &  6.0\spm{}0.0 &  6.0\spm{}0.0 &  6.0\spm{}0.0 &  1.0 &  1.0 &  1.0  \\
%   {\relsize{-1}hanoi(30)} &  12 &  12 &  12 &  12 &  12.0\spm{}0.0 &  12.0\spm{}0.0 &  12.0\spm{}0.0 &  12.0\spm{}0.0 &  12.0\spm{}0.0 &  1.0 &  1.0 &  1.0  \\
    {\relsize{-1}logistics00(28)} &  \textbf{20} &  \textbf{20} &  16 &  18 &  \textbf{20.0\spm{}0.0} &  \textbf{20.0\spm{}0.0} & \textbf{20.0\spm{}0.0} &  \textbf{20.0\spm{}0.0} &  \textbf{20.0\spm{}0.0} &  1.0 &  1.0 &  1.0  \\
    {\relsize{-1}miconic(150)} &  \textbf{140} & \textbf{140} &  68 &  \textbf{140} &  \textbf{140.0\spm{}0.0} & \textbf{140.0\spm{}0.0} & \textbf{140.0\spm{}0.0} &  135.6\spm{}0.5 &  \textbf{140.0\spm{}0.0} &  1.0 &  1.0 &  1.0  \\
    {\relsize{-1}mprime(35)} &  21 &  21 &  19 &  \textbf{22} &  20.9\spm{}0.3 &  20.9\spm{}0.3 &  20.9\spm{}0.3 &  21.0\spm{}0.0 &  20.9\spm{}0.3 &  1.0 &  1.0 &  1.0  \\
%   {\relsize{-1}mystery(30)} &  15 &  16 &  15 &  15 &  15.0\spm{}0.0 &  15.0\spm{}0.0 &  15.0\spm{}0.0 &  15.8\spm{}0.4 &  15.0\spm{}0.0 &  1.0 &  1.0 &  1.0  \\
    {\relsize{-1}nomystery-opt11(20)} &  \textbf{14} &  \textbf{14} &  12 &  13 &  \textbf{14.0\spm{}0.0} &  \textbf{14.0\spm{}0.0} &  \textbf{14.0\spm{}0.0} &  13.8\spm{}0.4 &  \textbf{14.0\spm{}0.0} &  1.0 &  1.0 &  1.0  \\
    {\relsize{-1}openstacks-opt11(20)} &  11 & \textbf{18} &  11 & \textbf{18} &  10.0\spm{}0.0 &  \textbf{18.0\spm{}0.0} &  \textbf{18.0\spm{}0.0} & \textbf{18.0\spm{}0.0} &  11.6\spm{}0.5 &  \textbf{0.0} &  1.0 &  \textbf{0.0}  \\
%   {\relsize{-1}parcprinter-opt11(20)} &  13 &  13 &  12 &  13 &  13.0\spm{}0.0 &  13.0\spm{}0.0 &  13.0\spm{}0.0 &  13.0\spm{}0.0 &  13.0\spm{}0.0 &  1.0 &  1.0 &  1.0  \\
%   {\relsize{-1}parking-opt11(20)} &  1 &  1 &  1 &  1 &  1.0\spm{}0.0 &  1.0\spm{}0.0 &  1.0\spm{}0.0 &  1.0\spm{}0.0 &  1.0\spm{}0.0 &  1.0 &  1.0 &  1.0  \\
%   {\relsize{-1}pathways(30)} &  5 &  5 &  4 &  5 &  5.0\spm{}0.0 &  5.0\spm{}0.0 &  5.0\spm{}0.0 &  5.0\spm{}0.0 &  5.0\spm{}0.0 &  1.0 &  1.0 &  1.0  \\
%   {\relsize{-1}pegsol-opt11(20)} &  17 &  17 &  17 &  17 &  17.0\spm{}0.0 &  17.0\spm{}0.0 &  17.0\spm{}0.0 &  17.0\spm{}0.0 &  17.0\spm{}0.0 &  1.0 &  1.0 &  1.0  \\
    {\relsize{-1}pipesworld-notankage(50)} &  \textbf{15} &  14 &  13 &  13 &  14.1\spm{}0.3 &  14.3\spm{}0.5 &  14.2\spm{}0.4 &  14.2\spm{}0.4 &  14.9\spm{}0.3 &  .58 &  .65 &  \textbf{0.0}  \\
%   {\relsize{-1}pipesworld-tankage(50)} &  8 &  8 &  7 &  8 &  8.0\spm{}0.0 &  8.0\spm{}0.0 &  8.0\spm{}0.0 &  8.0\spm{}0.0 &  8.0\spm{}0.0 &  1.0 &  1.0 &  1.0  \\
%   {\relsize{-1}psr-small(50)} &  48 &  48 &  48 &  48 &  48.0\spm{}0.0 &  48.0\spm{}0.0 &  48.0\spm{}0.0 &  48.0\spm{}0.0 &  48.0\spm{}0.0 &  1.0 &  1.0 &  1.0  \\
%   {\relsize{-1}rovers(40)} &  7 &  7 &  7 &  7 &  7.0\spm{}0.0 &  7.0\spm{}0.0 &  7.0\spm{}0.0 &  7.0\spm{}0.0 &  7.0\spm{}0.0 &  1.0 &  1.0 &  1.0  \\
    {\relsize{-1}scanalyzer-opt11(20)} &  \textbf{10} &  \textbf{10} &  4 &  \textbf{10} &  \textbf{10.0\spm{}0.0} &  \textbf{10.0\spm{}0.0} &  \textbf{10.0\spm{}0.0} &  9.0\spm{}0.0 & \textbf{10.0\spm{}0.0} &  1.0 &  1.0 &  1.0  \\
%   {\relsize{-1}sokoban-opt11(20)} &  19 &  19 &  19 &  19 &  19.0\spm{}0.0 &  19.0\spm{}0.0 &  19.0\spm{}0.0 &  19.0\spm{}0.0 &  19.0\spm{}0.0 &  1.0 &  1.0 &  1.0  \\
%   {\relsize{-1}storage(30)} &  14 &  14 &  14 &  14 &  14.0\spm{}0.0 &  14.0\spm{}0.0 &  14.0\spm{}0.0 &  14.4\spm{}0.5 &  14.0\spm{}0.0 &  1.0 &  1.0 &  1.0  \\
%   {\relsize{-1}tidybot-opt11(20)} &  12 &  12 &  11 &  11 &  12.0\spm{}0.0 &  12.0\spm{}0.0 &  12.0\spm{}0.0 &  11.8\spm{}0.4 &  12.0\spm{}0.0 &  1.0 &  1.0 &  1.0  \\
%   {\relsize{-1}tpp(30)} &  6 &  6 &  6 &  6 &  6.0\spm{}0.0 &  6.0\spm{}0.0 &  6.0\spm{}0.0 &  6.0\spm{}0.0 &  6.0\spm{}0.0 &  1.0 &  1.0 &  1.0  \\
%   {\relsize{-1}transport-opt11(20)} &  6 &  6 &  6 &  6 &  6.0\spm{}0.0 &  6.0\spm{}0.0 &  6.0\spm{}0.0 &  6.0\spm{}0.0 &  6.0\spm{}0.0 &  1.0 &  1.0 &  1.0  \\
%   {\relsize{-1}visitall-opt11(20)} &  10 &  10 &  9 &  10 &  10.0\spm{}0.0 &  10.0\spm{}0.0 &  10.0\spm{}0.0 &  10.0\spm{}0.0 &  10.0\spm{}0.0 &  1.0 &  1.0 &  1.0  \\
    {\relsize{-1}woodworking-opt11(20)} &  10 &  10 &  6 &  9 &  10.0\spm{}0.0 &  10.0\spm{}0.0 &  10.0\spm{}0.0 &  \textbf{11.8\spm{}0.4} &  10.0\spm{}0.0 &  1.0 &  1.0 &  1.0  \\
    {\relsize{-1}zenotravel(20)} &  \textbf{11} &  \textbf{11} &  9 &  \textbf{11} &  \textbf{11.0\spm{}0.0} &  \textbf{11.0\spm{}0.0} &  \textbf{11.0\spm{}0.0} & \textbf{11.0\spm{}0.0} & \textbf{11.0\spm{}0.0} &  1.0 &  1.0 &  1.0 \\\hline
    Zerocost(620) &  256 &  279 &  212 &  281 &  249.1\spm{}1.8 &  280.2\spm{}7.9 &  \textbf{287.2\spm{}2.4} &  280.2\spm{}4.2 &  264.9\spm{}1.8 &  \textbf{0.0} &  \textbf{.02} &  \textbf{0.0}  \\
  \hline                                    
    {\relsize{-1}airport-fuel(20)} &  \textbf{15} &  13 &  7 &  \textbf{15} &  14.2\spm{}0.9 &  13.8\spm{}0.6 &  14.4\spm{}0.7 &  10.4\spm{}0.5 &  14.4\spm{}0.7 &  .49 &  .06 &  1.0  \\
    {\relsize{-1}blocks-stack(20)} &  \textbf{17} &  \textbf{17} &  15 &  \textbf{17} &  \textbf{17.0\spm{}0.0} &  \textbf{17.1\spm{}0.3} &  \textbf{17.0\spm{}0.0} &  16.0\spm{}0.0 &  \textbf{17.0\spm{}0.0} &  1.0 &  .37 &  1.0  \\
%   {\relsize{-1}depot-fuel(22)} &  6 &  6 &  4 &  6 &  6.0\spm{}0.0 &  6.0\spm{}0.0 &  6.0\spm{}0.0 &  6.0\spm{}0.0 &  6.0\spm{}0.0 &  1.0 &  1.0 &  1.0  \\
%   {\relsize{-1}driverlog-fuel(20)} &  8 &  8 &  7 &  8 &  8.0\spm{}0.0 &  7.2\spm{}0.7 &  8.0\spm{}0.0 &  8.0\spm{}0.0 &  8.0\spm{}0.0 &  1.0 &  \textbf{.01} &  1.0  \\
    {\relsize{-1}elevators-up(20)} &  7 &  \textbf{13} &  7 &  \textbf{13} &  5.3\spm{}0.5 &  8.8\spm{}0.9 &  9.4\spm{}1.1 &  8.2\spm{}0.7 &  7.3\spm{}0.5 &  \textbf{0.0} &  .25 &  \textbf{0.0}  \\
%   {\relsize{-1}floortile-ink(20)} &  8 &  8 &  8 &  8 &  8.0\spm{}0.0 &  8.0\spm{}0.0 &  8.1\spm{}0.3 &  8.0\spm{}0.0 &  8.3\spm{}0.5 &  .37 &  .37 &  0.3  \\
    {\relsize{-1}freecell-move(20)} &  4 &  19 &  4 &  19 &  4.0\spm{}0.0 &  \textbf{19.4\spm{}0.5} &  16.5\spm{}0.7 &  16.6\spm{}0.8 &  5.0\spm{}0.4 &  \textbf{0.0} &  \textbf{0.0} &  \textbf{0.0}  \\
%   {\relsize{-1}grid-fuel(5)} &  1 &  1 &  1 &  1 &  1.0\spm{}0.0 &  1.0\spm{}0.0 &  1.0\spm{}0.0 &  1.0\spm{}0.0 &  1.0\spm{}0.0 &  1.0 &  1.0 &  1.0  \\
%   {\relsize{-1}gripper-move(20)} &  7 &  7 &  7 &  7 &  6.0\spm{}0.0 &  6.0\spm{}0.0 &  6.0\spm{}0.0 &  7.0\spm{}0.0 &  7.0\spm{}0.0 &  1.0 &  1.0 &  \textbf{0.0}  \\
%   {\relsize{-1}hiking-fuel(20)} &  9 &  9 &  8 &  9 &  9.0\spm{}0.0 &  9.0\spm{}0.0 &  9.0\spm{}0.0 &  9.0\spm{}0.0 &  9.0\spm{}0.0 &  1.0 &  1.0 &  1.0  \\
%   {\relsize{-1}logistics00-fuel(28)} &  16 &  16 &  15 &  16 &  15.0\spm{}0.0 &  15.0\spm{}0.0 &  15.0\spm{}0.0 &  16.0\spm{}0.0 &  16.0\spm{}0.0 &  1.0 &  1.0 &  \textbf{0.0}  \\
    {\relsize{-1}miconic-up(30)} &  16 &  17 &  10 &  17 &  15.4\spm{}0.5 &  18.0\spm{}1.3 &  19.8\spm{}1.0 &  \textbf{20.4\spm{}1.0} &  17.0\spm{}0.4 &  \textbf{0.0} &  \textbf{0.0} &  \textbf{0.0}  \\
    {\relsize{-1}mprime-succumb(35)} &  15 &  14 &  12 &  14 &  15.8\spm{}0.7 &  18.7\spm{}3.9 &  \textbf{20.1\spm{}0.7} &  18.6\spm{}2.0 &  17.9\spm{}0.5 &  \textbf{0.0} &  .23 &  \textbf{0.0}  \\
%   {\relsize{-1}mystery-feast(20)} &  7 &  5 &  5 &  5 &  7.2\spm{}0.4 &  6.2\spm{}0.7 &  7.2\spm{}0.4 &  7.2\spm{}0.7 &  7.3\spm{}0.5 &  1.0 &  \textbf{0.0} &  .65  \\
%   {\relsize{-1}nomystery-fuel(20)} &  10 &  10 &  9 &  10 &  10.0\spm{}0.0 &  10.0\spm{}0.0 &  10.0\spm{}0.0 &  9.4\spm{}0.5 &  10.0\spm{}0.0 &  1.0 &  1.0 &  1.0  \\
%   {\relsize{-1}parking-movecc(20)} &  0 &  0 &  0 &  0 &  0.0\spm{}0.0 &  0.0\spm{}0.0 &  0.0\spm{}0.0 &  0.0\spm{}0.0 &  0.0\spm{}0.0 &  1.0 &  1.0 &  1.0  \\
%   {\relsize{-1}pathways-fuel(30)} &  5 &  5 &  4 &  5 &  4.0\spm{}0.0 &  4.2\spm{}0.4 &  4.4\spm{}0.5 &  4.8\spm{}0.4 &  4.4\spm{}0.5 &  \textbf{.03} &  .37 &  1.0  \\
    {\relsize{-1}pipesnt-pushstart(20)} &  8 &  8 &  6 &  7 &  8.0\spm{}0.0 &  8.6\spm{}1.3 &  \textbf{9.8\spm{}0.4} &  \textbf{9.8\spm{}0.4} &  8.5\spm{}0.5 &  \textbf{0.0} &  \textbf{.04} &  \textbf{0.0}  \\
    {\relsize{-1}pipesworld-pushend(20)} &  3 &  4 &  2 &  4 &  3.0\spm{}0.0 &  4.2\spm{}1.0 &  4.5\spm{}0.8 &  \textbf{5.4\spm{}0.8} &  3.9\spm{}0.3 &  \textbf{0.0} &  0.5 &  \textbf{.05}  \\
%   {\relsize{-1}psr-small-open(20)} &  19 &  19 &  19 &  19 &  18.0\spm{}0.0 &  19.0\spm{}0.0 &  19.0\spm{}0.0 &  19.0\spm{}0.0 &  19.0\spm{}0.0 &  \textbf{0.0} &  1.0 &  1.0  \\
%   {\relsize{-1}rovers-fuel(40)} &  8 &  8 &  7 &  9 &  8.0\spm{}0.0 &  8.0\spm{}0.0 &  8.0\spm{}0.0 &  9.0\spm{}0.0 &  8.0\spm{}0.0 &  1.0 &  1.0 &  1.0  \\
    {\relsize{-1}scanalyzer-analyze(20)} &  9 &  9 &  3 &  9 &  \textbf{9.7\spm{}0.6} &  9.3\spm{}0.5 &  9.1\spm{}0.3 &  7.4\spm{}1.0 &  9.1\spm{}0.3 &  \textbf{.02} &  0.3 &  1.0  \\
%   {\relsize{-1}sokoban-pushgoal(20)} &  18 &  18 &  18 &  18 &  18.0\spm{}0.0 &  17.0\spm{}0.0 &  17.9\spm{}0.3 &  17.0\spm{}0.0 &  18.0\spm{}0.0 &  .37 &  \textbf{0.0} &  .37  \\
%   {\relsize{-1}storage-lift(20)} &  4 &  4 &  4 &  4 &  4.0\spm{}0.0 &  5.0\spm{}1.2 &  4.4\spm{}0.5 &  4.6\spm{}0.5 &  4.6\spm{}0.5 &  \textbf{.03} &  .26 &  .41  \\
%   {\relsize{-1}tidybot-motion(20)} &  16 &  16 &  14 &  16 &  16.0\spm{}0.0 &  16.0\spm{}0.0 &  16.0\spm{}0.0 &  15.6\spm{}0.5 &  16.0\spm{}0.0 &  1.0 &  1.0 &  1.0  \\
    {\relsize{-1}tpp-fuel(30)} &  8 &  \textbf{11} &  7 &  \textbf{11} &  7.0\spm{}0.0 &  \textbf{11.0\spm{}0.0} &  \textbf{11.0\spm{}0.0} &  \textbf{11.0\spm{}0.0} &  8.1\spm{}0.3 &  \textbf{0.0} &  1.0 &  \textbf{0.0}  \\
    {\relsize{-1}woodworking-cut(20)} &  5 &  7 &  2 &  7 &  4.5\spm{}0.5 &  6.7\spm{}0.5 &  \textbf{8.6\spm{}0.9} &  7.8\spm{}0.7 &  7.1\spm{}0.3 &  \textbf{0.0} &  \textbf{0.0} &  \textbf{0.0}  \\
%   {\relsize{-1}zenotravel-fuel(20)} &  7 &  7 &  7 &  7 &  7.0\spm{}0.0 &  7.0\spm{}0.0 &  7.0\spm{}0.0 &  7.0\spm{}0.0 &  7.0\spm{}0.0 &  1.0 &  1.0 &  1.0 \\
\hline
 Total(1724) &  814 &  844 &  654 &  837 &  803.7\spm{}2.2 &  848.5\spm{}8.9 &  \textbf{857.8\spm{}2.9} &  840.2\spm{}4.4 &  824.7\spm{}2.1 &  \textbf{0.0} &  \textbf{.01} &  \textbf{0.0} \\\hline
\end{tabular}

 \caption{
 Full version of the lower half of \reftbl{depth} showing 
 the experiments on the Zerocost instances using \lmcut heuritics.
 Each cell shows the coverage of the domain solved with 5 min, 2GB.
 As in the original \reftbl{depth}, we highlighted the best results in
 \textbf{boldface} only when the maximum pairwise coverage difference $\mit{MaxDiff}>2$.
 }
 \label{lmcut-zerocost-full}
 }
\end{table*}

\newpage
\section{Full Tables for \reftbl{depth} : Part 3, IPC domains, \mands}

\begin{table*}[htb]
 {
 \centering
 \begin{tabular}{|c|c|c|c|c|c|c|c|c|c||c|c|c|}
\hline
 & \multicolumn{4}{|c|}{Coverages}
 & \multicolumn{5}{|c||}{Coverages (mean$\pm$sd)}
 & \multicolumn{3}{|c|}{Wilcoxon $p$ vs $[h,\rd,\ro]$} \\
\hline                                    
 Domain &  $[h,\fifo]$ &  $[h,\lifo]$ &  $[\fifo]$ &  $[\lifo]$ &  $[h,\fd,\ro]$ &  $[h,\ld,\ro]$ &  $[h,\rd,\ro]$ &  $[\rd,\ro]$ &  $[h,\ro]$ & $[h,\fd,\ro]$   & $[h,\ld,\ro]$   & $[h,\ro]$    \\
\hline                                    
 IPC Benchmark(1104) &  558 &  565 &  442 &  556 &  554.6\spm{}0.8 &  568.3\spm{}1.8 &  \textbf{570.6\spm{}1.5} &  560.0\spm{}0.9 &  559.8\spm{}1.0 &  \textbf{0.0} &  \textbf{.01} &  \textbf{0.0}  \\
\hline                                    
    {\relsize{-1}airport(50)} &  \textbf{27} &  26 &  18 &  26 &  25.6\spm{}0.5 &  25.8\spm{}0.6 &  25.9\spm{}0.5 &  21.0\spm{}0.0 &  26.0\spm{}0.0 &  .26 &  .72 &  .58  \\
%   {\relsize{-1}barman-opt11(20)} &  0 &  0 &  0 &  0 &  0.0\spm{}0.0 &  0.0\spm{}0.0 &  0.0\spm{}0.0 &  0.0\spm{}0.0 &  0.0\spm{}0.0 &  1.0 &  1.0 &  1.0  \\
%   {\relsize{-1}blocks(35)} &  28 &  28 &  26 &  26 &  28.0\spm{}0.0 &  28.0\spm{}0.0 &  28.0\spm{}0.0 &  27.0\spm{}0.0 &  28.0\spm{}0.0 &  1.0 &  1.0 &  1.0  \\
    {\relsize{-1}cybersec(19)} &  2 &  3 &  0 &  3 &  2.0\spm{}0.0 &  7.3\spm{}1.5 &  \textbf{9.6\spm{}1.1} &  7.8\spm{}0.7 &  4.4\spm{}1.0 &  \textbf{0.0} &  \textbf{.01} &  \textbf{0.0}  \\
%   {\relsize{-1}depot(22)} &  6 &  6 &  5 &  5 &  6.0\spm{}0.0 &  6.0\spm{}0.0 &  6.0\spm{}0.0 &  6.0\spm{}0.0 &  6.0\spm{}0.0 &  1.0 &  1.0 &  1.0  \\
%   {\relsize{-1}driverlog(20)} &  13 &  13 &  12 &  13 &  13.0\spm{}0.0 &  13.0\spm{}0.0 &  13.0\spm{}0.0 &  13.0\spm{}0.0 &  13.0\spm{}0.0 &  1.0 &  1.0 &  1.0  \\
%   {\relsize{-1}elevators-opt11(20)} &  15 &  15 &  14 &  15 &  15.0\spm{}0.0 &  15.0\spm{}0.0 &  15.0\spm{}0.0 &  14.8\spm{}0.4 &  15.0\spm{}0.0 &  1.0 &  1.0 &  1.0  \\
%   {\relsize{-1}floortile-opt11(20)} &  6 &  6 &  6 &  6 &  6.0\spm{}0.0 &  6.0\spm{}0.0 &  6.0\spm{}0.0 &  6.0\spm{}0.0 &  6.0\spm{}0.0 &  1.0 &  1.0 &  1.0  \\
%   {\relsize{-1}freecell(80)} &  9 &  9 &  8 &  9 &  9.0\spm{}0.0 &  9.0\spm{}0.0 &  9.0\spm{}0.0 &  9.0\spm{}0.0 &  9.0\spm{}0.0 &  1.0 &  1.0 &  1.0  \\
%   {\relsize{-1}grid(5)} &  1 &  1 &  1 &  1 &  1.0\spm{}0.0 &  1.0\spm{}0.0 &  1.0\spm{}0.0 &  1.0\spm{}0.0 &  1.0\spm{}0.0 &  1.0 &  1.0 &  1.0  \\
%   {\relsize{-1}gripper(20)} &  6 &  6 &  6 &  6 &  6.0\spm{}0.0 &  6.0\spm{}0.0 &  6.0\spm{}0.0 &  6.0\spm{}0.0 &  6.0\spm{}0.0 &  1.0 &  1.0 &  1.0  \\
%   {\relsize{-1}hanoi(30)} &  12 &  12 &  12 &  12 &  12.0\spm{}0.0 &  12.0\spm{}0.0 &  12.0\spm{}0.0 &  12.0\spm{}0.0 &  12.0\spm{}0.0 &  1.0 &  1.0 &  1.0  \\
    {\relsize{-1}logistics00(28)} &  \textbf{20} &  \textbf{20} &  16 &  18 &  \textbf{20.0\spm{}0.0} &  \textbf{20.0\spm{}0.0} & \textbf{20.0\spm{}0.0} &  \textbf{20.0\spm{}0.0} &  \textbf{20.0\spm{}0.0} &  1.0 &  1.0 &  1.0  \\
    {\relsize{-1}miconic(150)} &  \textbf{140} & \textbf{140} &  68 &  \textbf{140} &  \textbf{140.0\spm{}0.0} & \textbf{140.0\spm{}0.0} & \textbf{140.0\spm{}0.0} &  135.6\spm{}0.5 &  \textbf{140.0\spm{}0.0} &  1.0 &  1.0 &  1.0  \\
    {\relsize{-1}mprime(35)} &  21 &  21 &  19 &  \textbf{22} &  20.9\spm{}0.3 &  20.9\spm{}0.3 &  20.9\spm{}0.3 &  21.0\spm{}0.0 &  20.9\spm{}0.3 &  1.0 &  1.0 &  1.0  \\
%   {\relsize{-1}mystery(30)} &  15 &  16 &  15 &  15 &  15.0\spm{}0.0 &  15.0\spm{}0.0 &  15.0\spm{}0.0 &  15.8\spm{}0.4 &  15.0\spm{}0.0 &  1.0 &  1.0 &  1.0  \\
    {\relsize{-1}nomystery-opt11(20)} &  \textbf{14} &  \textbf{14} &  12 &  13 &  \textbf{14.0\spm{}0.0} &  \textbf{14.0\spm{}0.0} &  \textbf{14.0\spm{}0.0} &  13.8\spm{}0.4 &  \textbf{14.0\spm{}0.0} &  1.0 &  1.0 &  1.0  \\
    {\relsize{-1}openstacks-opt11(20)} &  11 & \textbf{18} &  11 & \textbf{18} &  10.0\spm{}0.0 &  \textbf{18.0\spm{}0.0} &  \textbf{18.0\spm{}0.0} & \textbf{18.0\spm{}0.0} &  11.6\spm{}0.5 &  \textbf{0.0} &  1.0 &  \textbf{0.0}  \\
%   {\relsize{-1}parcprinter-opt11(20)} &  13 &  13 &  12 &  13 &  13.0\spm{}0.0 &  13.0\spm{}0.0 &  13.0\spm{}0.0 &  13.0\spm{}0.0 &  13.0\spm{}0.0 &  1.0 &  1.0 &  1.0  \\
%   {\relsize{-1}parking-opt11(20)} &  1 &  1 &  1 &  1 &  1.0\spm{}0.0 &  1.0\spm{}0.0 &  1.0\spm{}0.0 &  1.0\spm{}0.0 &  1.0\spm{}0.0 &  1.0 &  1.0 &  1.0  \\
%   {\relsize{-1}pathways(30)} &  5 &  5 &  4 &  5 &  5.0\spm{}0.0 &  5.0\spm{}0.0 &  5.0\spm{}0.0 &  5.0\spm{}0.0 &  5.0\spm{}0.0 &  1.0 &  1.0 &  1.0  \\
%   {\relsize{-1}pegsol-opt11(20)} &  17 &  17 &  17 &  17 &  17.0\spm{}0.0 &  17.0\spm{}0.0 &  17.0\spm{}0.0 &  17.0\spm{}0.0 &  17.0\spm{}0.0 &  1.0 &  1.0 &  1.0  \\
    {\relsize{-1}pipesworld-notankage(50)} &  \textbf{15} &  14 &  13 &  13 &  14.1\spm{}0.3 &  14.3\spm{}0.5 &  14.2\spm{}0.4 &  14.2\spm{}0.4 &  14.9\spm{}0.3 &  .58 &  .65 &  \textbf{0.0}  \\
%   {\relsize{-1}pipesworld-tankage(50)} &  8 &  8 &  7 &  8 &  8.0\spm{}0.0 &  8.0\spm{}0.0 &  8.0\spm{}0.0 &  8.0\spm{}0.0 &  8.0\spm{}0.0 &  1.0 &  1.0 &  1.0  \\
%   {\relsize{-1}psr-small(50)} &  48 &  48 &  48 &  48 &  48.0\spm{}0.0 &  48.0\spm{}0.0 &  48.0\spm{}0.0 &  48.0\spm{}0.0 &  48.0\spm{}0.0 &  1.0 &  1.0 &  1.0  \\
%   {\relsize{-1}rovers(40)} &  7 &  7 &  7 &  7 &  7.0\spm{}0.0 &  7.0\spm{}0.0 &  7.0\spm{}0.0 &  7.0\spm{}0.0 &  7.0\spm{}0.0 &  1.0 &  1.0 &  1.0  \\
    {\relsize{-1}scanalyzer-opt11(20)} &  \textbf{10} &  \textbf{10} &  4 &  \textbf{10} &  \textbf{10.0\spm{}0.0} &  \textbf{10.0\spm{}0.0} &  \textbf{10.0\spm{}0.0} &  9.0\spm{}0.0 & \textbf{10.0\spm{}0.0} &  1.0 &  1.0 &  1.0  \\
%   {\relsize{-1}sokoban-opt11(20)} &  19 &  19 &  19 &  19 &  19.0\spm{}0.0 &  19.0\spm{}0.0 &  19.0\spm{}0.0 &  19.0\spm{}0.0 &  19.0\spm{}0.0 &  1.0 &  1.0 &  1.0  \\
%   {\relsize{-1}storage(30)} &  14 &  14 &  14 &  14 &  14.0\spm{}0.0 &  14.0\spm{}0.0 &  14.0\spm{}0.0 &  14.4\spm{}0.5 &  14.0\spm{}0.0 &  1.0 &  1.0 &  1.0  \\
%   {\relsize{-1}tidybot-opt11(20)} &  12 &  12 &  11 &  11 &  12.0\spm{}0.0 &  12.0\spm{}0.0 &  12.0\spm{}0.0 &  11.8\spm{}0.4 &  12.0\spm{}0.0 &  1.0 &  1.0 &  1.0  \\
%   {\relsize{-1}tpp(30)} &  6 &  6 &  6 &  6 &  6.0\spm{}0.0 &  6.0\spm{}0.0 &  6.0\spm{}0.0 &  6.0\spm{}0.0 &  6.0\spm{}0.0 &  1.0 &  1.0 &  1.0  \\
%   {\relsize{-1}transport-opt11(20)} &  6 &  6 &  6 &  6 &  6.0\spm{}0.0 &  6.0\spm{}0.0 &  6.0\spm{}0.0 &  6.0\spm{}0.0 &  6.0\spm{}0.0 &  1.0 &  1.0 &  1.0  \\
%   {\relsize{-1}visitall-opt11(20)} &  10 &  10 &  9 &  10 &  10.0\spm{}0.0 &  10.0\spm{}0.0 &  10.0\spm{}0.0 &  10.0\spm{}0.0 &  10.0\spm{}0.0 &  1.0 &  1.0 &  1.0  \\
    {\relsize{-1}woodworking-opt11(20)} &  10 &  10 &  6 &  9 &  10.0\spm{}0.0 &  10.0\spm{}0.0 &  10.0\spm{}0.0 &  \textbf{11.8\spm{}0.4} &  10.0\spm{}0.0 &  1.0 &  1.0 &  1.0  \\
    {\relsize{-1}zenotravel(20)} &  \textbf{11} &  \textbf{11} &  9 &  \textbf{11} &  \textbf{11.0\spm{}0.0} &  \textbf{11.0\spm{}0.0} &  \textbf{11.0\spm{}0.0} & \textbf{11.0\spm{}0.0} & \textbf{11.0\spm{}0.0} &  1.0 &  1.0 &  1.0 \\\hline
    Zerocost(620) &  256 &  279 &  212 &  281 &  249.1\spm{}1.8 &  280.2\spm{}7.9 &  \textbf{287.2\spm{}2.4} &  280.2\spm{}4.2 &  264.9\spm{}1.8 &  \textbf{0.0} &  \textbf{.02} &  \textbf{0.0}  \\
  \hline                                    
    {\relsize{-1}airport-fuel(20)} &  \textbf{15} &  13 &  7 &  \textbf{15} &  14.2\spm{}0.9 &  13.8\spm{}0.6 &  14.4\spm{}0.7 &  10.4\spm{}0.5 &  14.4\spm{}0.7 &  .49 &  .06 &  1.0  \\
    {\relsize{-1}blocks-stack(20)} &  \textbf{17} &  \textbf{17} &  15 &  \textbf{17} &  \textbf{17.0\spm{}0.0} &  \textbf{17.1\spm{}0.3} &  \textbf{17.0\spm{}0.0} &  16.0\spm{}0.0 &  \textbf{17.0\spm{}0.0} &  1.0 &  .37 &  1.0  \\
%   {\relsize{-1}depot-fuel(22)} &  6 &  6 &  4 &  6 &  6.0\spm{}0.0 &  6.0\spm{}0.0 &  6.0\spm{}0.0 &  6.0\spm{}0.0 &  6.0\spm{}0.0 &  1.0 &  1.0 &  1.0  \\
%   {\relsize{-1}driverlog-fuel(20)} &  8 &  8 &  7 &  8 &  8.0\spm{}0.0 &  7.2\spm{}0.7 &  8.0\spm{}0.0 &  8.0\spm{}0.0 &  8.0\spm{}0.0 &  1.0 &  \textbf{.01} &  1.0  \\
    {\relsize{-1}elevators-up(20)} &  7 &  \textbf{13} &  7 &  \textbf{13} &  5.3\spm{}0.5 &  8.8\spm{}0.9 &  9.4\spm{}1.1 &  8.2\spm{}0.7 &  7.3\spm{}0.5 &  \textbf{0.0} &  .25 &  \textbf{0.0}  \\
%   {\relsize{-1}floortile-ink(20)} &  8 &  8 &  8 &  8 &  8.0\spm{}0.0 &  8.0\spm{}0.0 &  8.1\spm{}0.3 &  8.0\spm{}0.0 &  8.3\spm{}0.5 &  .37 &  .37 &  0.3  \\
    {\relsize{-1}freecell-move(20)} &  4 &  19 &  4 &  19 &  4.0\spm{}0.0 &  \textbf{19.4\spm{}0.5} &  16.5\spm{}0.7 &  16.6\spm{}0.8 &  5.0\spm{}0.4 &  \textbf{0.0} &  \textbf{0.0} &  \textbf{0.0}  \\
%   {\relsize{-1}grid-fuel(5)} &  1 &  1 &  1 &  1 &  1.0\spm{}0.0 &  1.0\spm{}0.0 &  1.0\spm{}0.0 &  1.0\spm{}0.0 &  1.0\spm{}0.0 &  1.0 &  1.0 &  1.0  \\
%   {\relsize{-1}gripper-move(20)} &  7 &  7 &  7 &  7 &  6.0\spm{}0.0 &  6.0\spm{}0.0 &  6.0\spm{}0.0 &  7.0\spm{}0.0 &  7.0\spm{}0.0 &  1.0 &  1.0 &  \textbf{0.0}  \\
%   {\relsize{-1}hiking-fuel(20)} &  9 &  9 &  8 &  9 &  9.0\spm{}0.0 &  9.0\spm{}0.0 &  9.0\spm{}0.0 &  9.0\spm{}0.0 &  9.0\spm{}0.0 &  1.0 &  1.0 &  1.0  \\
%   {\relsize{-1}logistics00-fuel(28)} &  16 &  16 &  15 &  16 &  15.0\spm{}0.0 &  15.0\spm{}0.0 &  15.0\spm{}0.0 &  16.0\spm{}0.0 &  16.0\spm{}0.0 &  1.0 &  1.0 &  \textbf{0.0}  \\
    {\relsize{-1}miconic-up(30)} &  16 &  17 &  10 &  17 &  15.4\spm{}0.5 &  18.0\spm{}1.3 &  19.8\spm{}1.0 &  \textbf{20.4\spm{}1.0} &  17.0\spm{}0.4 &  \textbf{0.0} &  \textbf{0.0} &  \textbf{0.0}  \\
    {\relsize{-1}mprime-succumb(35)} &  15 &  14 &  12 &  14 &  15.8\spm{}0.7 &  18.7\spm{}3.9 &  \textbf{20.1\spm{}0.7} &  18.6\spm{}2.0 &  17.9\spm{}0.5 &  \textbf{0.0} &  .23 &  \textbf{0.0}  \\
%   {\relsize{-1}mystery-feast(20)} &  7 &  5 &  5 &  5 &  7.2\spm{}0.4 &  6.2\spm{}0.7 &  7.2\spm{}0.4 &  7.2\spm{}0.7 &  7.3\spm{}0.5 &  1.0 &  \textbf{0.0} &  .65  \\
%   {\relsize{-1}nomystery-fuel(20)} &  10 &  10 &  9 &  10 &  10.0\spm{}0.0 &  10.0\spm{}0.0 &  10.0\spm{}0.0 &  9.4\spm{}0.5 &  10.0\spm{}0.0 &  1.0 &  1.0 &  1.0  \\
%   {\relsize{-1}parking-movecc(20)} &  0 &  0 &  0 &  0 &  0.0\spm{}0.0 &  0.0\spm{}0.0 &  0.0\spm{}0.0 &  0.0\spm{}0.0 &  0.0\spm{}0.0 &  1.0 &  1.0 &  1.0  \\
%   {\relsize{-1}pathways-fuel(30)} &  5 &  5 &  4 &  5 &  4.0\spm{}0.0 &  4.2\spm{}0.4 &  4.4\spm{}0.5 &  4.8\spm{}0.4 &  4.4\spm{}0.5 &  \textbf{.03} &  .37 &  1.0  \\
    {\relsize{-1}pipesnt-pushstart(20)} &  8 &  8 &  6 &  7 &  8.0\spm{}0.0 &  8.6\spm{}1.3 &  \textbf{9.8\spm{}0.4} &  \textbf{9.8\spm{}0.4} &  8.5\spm{}0.5 &  \textbf{0.0} &  \textbf{.04} &  \textbf{0.0}  \\
    {\relsize{-1}pipesworld-pushend(20)} &  3 &  4 &  2 &  4 &  3.0\spm{}0.0 &  4.2\spm{}1.0 &  4.5\spm{}0.8 &  \textbf{5.4\spm{}0.8} &  3.9\spm{}0.3 &  \textbf{0.0} &  0.5 &  \textbf{.05}  \\
%   {\relsize{-1}psr-small-open(20)} &  19 &  19 &  19 &  19 &  18.0\spm{}0.0 &  19.0\spm{}0.0 &  19.0\spm{}0.0 &  19.0\spm{}0.0 &  19.0\spm{}0.0 &  \textbf{0.0} &  1.0 &  1.0  \\
%   {\relsize{-1}rovers-fuel(40)} &  8 &  8 &  7 &  9 &  8.0\spm{}0.0 &  8.0\spm{}0.0 &  8.0\spm{}0.0 &  9.0\spm{}0.0 &  8.0\spm{}0.0 &  1.0 &  1.0 &  1.0  \\
    {\relsize{-1}scanalyzer-analyze(20)} &  9 &  9 &  3 &  9 &  \textbf{9.7\spm{}0.6} &  9.3\spm{}0.5 &  9.1\spm{}0.3 &  7.4\spm{}1.0 &  9.1\spm{}0.3 &  \textbf{.02} &  0.3 &  1.0  \\
%   {\relsize{-1}sokoban-pushgoal(20)} &  18 &  18 &  18 &  18 &  18.0\spm{}0.0 &  17.0\spm{}0.0 &  17.9\spm{}0.3 &  17.0\spm{}0.0 &  18.0\spm{}0.0 &  .37 &  \textbf{0.0} &  .37  \\
%   {\relsize{-1}storage-lift(20)} &  4 &  4 &  4 &  4 &  4.0\spm{}0.0 &  5.0\spm{}1.2 &  4.4\spm{}0.5 &  4.6\spm{}0.5 &  4.6\spm{}0.5 &  \textbf{.03} &  .26 &  .41  \\
%   {\relsize{-1}tidybot-motion(20)} &  16 &  16 &  14 &  16 &  16.0\spm{}0.0 &  16.0\spm{}0.0 &  16.0\spm{}0.0 &  15.6\spm{}0.5 &  16.0\spm{}0.0 &  1.0 &  1.0 &  1.0  \\
    {\relsize{-1}tpp-fuel(30)} &  8 &  \textbf{11} &  7 &  \textbf{11} &  7.0\spm{}0.0 &  \textbf{11.0\spm{}0.0} &  \textbf{11.0\spm{}0.0} &  \textbf{11.0\spm{}0.0} &  8.1\spm{}0.3 &  \textbf{0.0} &  1.0 &  \textbf{0.0}  \\
    {\relsize{-1}woodworking-cut(20)} &  5 &  7 &  2 &  7 &  4.5\spm{}0.5 &  6.7\spm{}0.5 &  \textbf{8.6\spm{}0.9} &  7.8\spm{}0.7 &  7.1\spm{}0.3 &  \textbf{0.0} &  \textbf{0.0} &  \textbf{0.0}  \\
%   {\relsize{-1}zenotravel-fuel(20)} &  7 &  7 &  7 &  7 &  7.0\spm{}0.0 &  7.0\spm{}0.0 &  7.0\spm{}0.0 &  7.0\spm{}0.0 &  7.0\spm{}0.0 &  1.0 &  1.0 &  1.0 \\
\hline
 Total(1724) &  814 &  844 &  654 &  837 &  803.7\spm{}2.2 &  848.5\spm{}8.9 &  \textbf{857.8\spm{}2.9} &  840.2\spm{}4.4 &  824.7\spm{}2.1 &  \textbf{0.0} &  \textbf{.01} &  \textbf{0.0} \\\hline
\end{tabular}

 \caption{
 Full version of the bottom line of \reftbl{depth} showing 
 the experiments on the IPC benchmark instances using \mands heuritics.
 Each cell shows the coverage of the domain solved with 5 min, 2GB.
 As in the original \reftbl{depth}, we highlighted the best results in
 \textbf{boldface} only when the maximum pairwise coverage difference $\mit{MaxDiff}>2$.
 }
 \label{mands-ipc-full}
 }
\end{table*}

\newpage
\section{Full Tables for \reftbl{depth} : Part 4, Zerocost domains, \mands}

\begin{table*}[htb]
 {
 \centering
 \begin{tabular}{|c|c|c|c|c|c|c|c|c|c||c|c|c|}
\hline
 & \multicolumn{4}{|c|}{Coverages}
 & \multicolumn{5}{|c||}{Coverages (mean$\pm$sd)}
 & \multicolumn{3}{|c|}{Wilcoxon $p$ vs $[h,\rd,\ro]$} \\
\hline                                    
 Domain &  $[h,\fifo]$ &  $[h,\lifo]$ &  $[\fifo]$ &  $[\lifo]$ &  $[h,\fd,\ro]$ &  $[h,\ld,\ro]$ &  $[h,\rd,\ro]$ &  $[\rd,\ro]$ &  $[h,\ro]$ & $[h,\fd,\ro]$   & $[h,\ld,\ro]$   & $[h,\ro]$    \\
\hline                                    
 IPC Benchmark(1104) &  558 &  565 &  442 &  556 &  554.6\spm{}0.8 &  568.3\spm{}1.8 &  \textbf{570.6\spm{}1.5} &  560.0\spm{}0.9 &  559.8\spm{}1.0 &  \textbf{0.0} &  \textbf{.01} &  \textbf{0.0}  \\
\hline                                    
    {\relsize{-1}airport(50)} &  \textbf{27} &  26 &  18 &  26 &  25.6\spm{}0.5 &  25.8\spm{}0.6 &  25.9\spm{}0.5 &  21.0\spm{}0.0 &  26.0\spm{}0.0 &  .26 &  .72 &  .58  \\
%   {\relsize{-1}barman-opt11(20)} &  0 &  0 &  0 &  0 &  0.0\spm{}0.0 &  0.0\spm{}0.0 &  0.0\spm{}0.0 &  0.0\spm{}0.0 &  0.0\spm{}0.0 &  1.0 &  1.0 &  1.0  \\
%   {\relsize{-1}blocks(35)} &  28 &  28 &  26 &  26 &  28.0\spm{}0.0 &  28.0\spm{}0.0 &  28.0\spm{}0.0 &  27.0\spm{}0.0 &  28.0\spm{}0.0 &  1.0 &  1.0 &  1.0  \\
    {\relsize{-1}cybersec(19)} &  2 &  3 &  0 &  3 &  2.0\spm{}0.0 &  7.3\spm{}1.5 &  \textbf{9.6\spm{}1.1} &  7.8\spm{}0.7 &  4.4\spm{}1.0 &  \textbf{0.0} &  \textbf{.01} &  \textbf{0.0}  \\
%   {\relsize{-1}depot(22)} &  6 &  6 &  5 &  5 &  6.0\spm{}0.0 &  6.0\spm{}0.0 &  6.0\spm{}0.0 &  6.0\spm{}0.0 &  6.0\spm{}0.0 &  1.0 &  1.0 &  1.0  \\
%   {\relsize{-1}driverlog(20)} &  13 &  13 &  12 &  13 &  13.0\spm{}0.0 &  13.0\spm{}0.0 &  13.0\spm{}0.0 &  13.0\spm{}0.0 &  13.0\spm{}0.0 &  1.0 &  1.0 &  1.0  \\
%   {\relsize{-1}elevators-opt11(20)} &  15 &  15 &  14 &  15 &  15.0\spm{}0.0 &  15.0\spm{}0.0 &  15.0\spm{}0.0 &  14.8\spm{}0.4 &  15.0\spm{}0.0 &  1.0 &  1.0 &  1.0  \\
%   {\relsize{-1}floortile-opt11(20)} &  6 &  6 &  6 &  6 &  6.0\spm{}0.0 &  6.0\spm{}0.0 &  6.0\spm{}0.0 &  6.0\spm{}0.0 &  6.0\spm{}0.0 &  1.0 &  1.0 &  1.0  \\
%   {\relsize{-1}freecell(80)} &  9 &  9 &  8 &  9 &  9.0\spm{}0.0 &  9.0\spm{}0.0 &  9.0\spm{}0.0 &  9.0\spm{}0.0 &  9.0\spm{}0.0 &  1.0 &  1.0 &  1.0  \\
%   {\relsize{-1}grid(5)} &  1 &  1 &  1 &  1 &  1.0\spm{}0.0 &  1.0\spm{}0.0 &  1.0\spm{}0.0 &  1.0\spm{}0.0 &  1.0\spm{}0.0 &  1.0 &  1.0 &  1.0  \\
%   {\relsize{-1}gripper(20)} &  6 &  6 &  6 &  6 &  6.0\spm{}0.0 &  6.0\spm{}0.0 &  6.0\spm{}0.0 &  6.0\spm{}0.0 &  6.0\spm{}0.0 &  1.0 &  1.0 &  1.0  \\
%   {\relsize{-1}hanoi(30)} &  12 &  12 &  12 &  12 &  12.0\spm{}0.0 &  12.0\spm{}0.0 &  12.0\spm{}0.0 &  12.0\spm{}0.0 &  12.0\spm{}0.0 &  1.0 &  1.0 &  1.0  \\
    {\relsize{-1}logistics00(28)} &  \textbf{20} &  \textbf{20} &  16 &  18 &  \textbf{20.0\spm{}0.0} &  \textbf{20.0\spm{}0.0} & \textbf{20.0\spm{}0.0} &  \textbf{20.0\spm{}0.0} &  \textbf{20.0\spm{}0.0} &  1.0 &  1.0 &  1.0  \\
    {\relsize{-1}miconic(150)} &  \textbf{140} & \textbf{140} &  68 &  \textbf{140} &  \textbf{140.0\spm{}0.0} & \textbf{140.0\spm{}0.0} & \textbf{140.0\spm{}0.0} &  135.6\spm{}0.5 &  \textbf{140.0\spm{}0.0} &  1.0 &  1.0 &  1.0  \\
    {\relsize{-1}mprime(35)} &  21 &  21 &  19 &  \textbf{22} &  20.9\spm{}0.3 &  20.9\spm{}0.3 &  20.9\spm{}0.3 &  21.0\spm{}0.0 &  20.9\spm{}0.3 &  1.0 &  1.0 &  1.0  \\
%   {\relsize{-1}mystery(30)} &  15 &  16 &  15 &  15 &  15.0\spm{}0.0 &  15.0\spm{}0.0 &  15.0\spm{}0.0 &  15.8\spm{}0.4 &  15.0\spm{}0.0 &  1.0 &  1.0 &  1.0  \\
    {\relsize{-1}nomystery-opt11(20)} &  \textbf{14} &  \textbf{14} &  12 &  13 &  \textbf{14.0\spm{}0.0} &  \textbf{14.0\spm{}0.0} &  \textbf{14.0\spm{}0.0} &  13.8\spm{}0.4 &  \textbf{14.0\spm{}0.0} &  1.0 &  1.0 &  1.0  \\
    {\relsize{-1}openstacks-opt11(20)} &  11 & \textbf{18} &  11 & \textbf{18} &  10.0\spm{}0.0 &  \textbf{18.0\spm{}0.0} &  \textbf{18.0\spm{}0.0} & \textbf{18.0\spm{}0.0} &  11.6\spm{}0.5 &  \textbf{0.0} &  1.0 &  \textbf{0.0}  \\
%   {\relsize{-1}parcprinter-opt11(20)} &  13 &  13 &  12 &  13 &  13.0\spm{}0.0 &  13.0\spm{}0.0 &  13.0\spm{}0.0 &  13.0\spm{}0.0 &  13.0\spm{}0.0 &  1.0 &  1.0 &  1.0  \\
%   {\relsize{-1}parking-opt11(20)} &  1 &  1 &  1 &  1 &  1.0\spm{}0.0 &  1.0\spm{}0.0 &  1.0\spm{}0.0 &  1.0\spm{}0.0 &  1.0\spm{}0.0 &  1.0 &  1.0 &  1.0  \\
%   {\relsize{-1}pathways(30)} &  5 &  5 &  4 &  5 &  5.0\spm{}0.0 &  5.0\spm{}0.0 &  5.0\spm{}0.0 &  5.0\spm{}0.0 &  5.0\spm{}0.0 &  1.0 &  1.0 &  1.0  \\
%   {\relsize{-1}pegsol-opt11(20)} &  17 &  17 &  17 &  17 &  17.0\spm{}0.0 &  17.0\spm{}0.0 &  17.0\spm{}0.0 &  17.0\spm{}0.0 &  17.0\spm{}0.0 &  1.0 &  1.0 &  1.0  \\
    {\relsize{-1}pipesworld-notankage(50)} &  \textbf{15} &  14 &  13 &  13 &  14.1\spm{}0.3 &  14.3\spm{}0.5 &  14.2\spm{}0.4 &  14.2\spm{}0.4 &  14.9\spm{}0.3 &  .58 &  .65 &  \textbf{0.0}  \\
%   {\relsize{-1}pipesworld-tankage(50)} &  8 &  8 &  7 &  8 &  8.0\spm{}0.0 &  8.0\spm{}0.0 &  8.0\spm{}0.0 &  8.0\spm{}0.0 &  8.0\spm{}0.0 &  1.0 &  1.0 &  1.0  \\
%   {\relsize{-1}psr-small(50)} &  48 &  48 &  48 &  48 &  48.0\spm{}0.0 &  48.0\spm{}0.0 &  48.0\spm{}0.0 &  48.0\spm{}0.0 &  48.0\spm{}0.0 &  1.0 &  1.0 &  1.0  \\
%   {\relsize{-1}rovers(40)} &  7 &  7 &  7 &  7 &  7.0\spm{}0.0 &  7.0\spm{}0.0 &  7.0\spm{}0.0 &  7.0\spm{}0.0 &  7.0\spm{}0.0 &  1.0 &  1.0 &  1.0  \\
    {\relsize{-1}scanalyzer-opt11(20)} &  \textbf{10} &  \textbf{10} &  4 &  \textbf{10} &  \textbf{10.0\spm{}0.0} &  \textbf{10.0\spm{}0.0} &  \textbf{10.0\spm{}0.0} &  9.0\spm{}0.0 & \textbf{10.0\spm{}0.0} &  1.0 &  1.0 &  1.0  \\
%   {\relsize{-1}sokoban-opt11(20)} &  19 &  19 &  19 &  19 &  19.0\spm{}0.0 &  19.0\spm{}0.0 &  19.0\spm{}0.0 &  19.0\spm{}0.0 &  19.0\spm{}0.0 &  1.0 &  1.0 &  1.0  \\
%   {\relsize{-1}storage(30)} &  14 &  14 &  14 &  14 &  14.0\spm{}0.0 &  14.0\spm{}0.0 &  14.0\spm{}0.0 &  14.4\spm{}0.5 &  14.0\spm{}0.0 &  1.0 &  1.0 &  1.0  \\
%   {\relsize{-1}tidybot-opt11(20)} &  12 &  12 &  11 &  11 &  12.0\spm{}0.0 &  12.0\spm{}0.0 &  12.0\spm{}0.0 &  11.8\spm{}0.4 &  12.0\spm{}0.0 &  1.0 &  1.0 &  1.0  \\
%   {\relsize{-1}tpp(30)} &  6 &  6 &  6 &  6 &  6.0\spm{}0.0 &  6.0\spm{}0.0 &  6.0\spm{}0.0 &  6.0\spm{}0.0 &  6.0\spm{}0.0 &  1.0 &  1.0 &  1.0  \\
%   {\relsize{-1}transport-opt11(20)} &  6 &  6 &  6 &  6 &  6.0\spm{}0.0 &  6.0\spm{}0.0 &  6.0\spm{}0.0 &  6.0\spm{}0.0 &  6.0\spm{}0.0 &  1.0 &  1.0 &  1.0  \\
%   {\relsize{-1}visitall-opt11(20)} &  10 &  10 &  9 &  10 &  10.0\spm{}0.0 &  10.0\spm{}0.0 &  10.0\spm{}0.0 &  10.0\spm{}0.0 &  10.0\spm{}0.0 &  1.0 &  1.0 &  1.0  \\
    {\relsize{-1}woodworking-opt11(20)} &  10 &  10 &  6 &  9 &  10.0\spm{}0.0 &  10.0\spm{}0.0 &  10.0\spm{}0.0 &  \textbf{11.8\spm{}0.4} &  10.0\spm{}0.0 &  1.0 &  1.0 &  1.0  \\
    {\relsize{-1}zenotravel(20)} &  \textbf{11} &  \textbf{11} &  9 &  \textbf{11} &  \textbf{11.0\spm{}0.0} &  \textbf{11.0\spm{}0.0} &  \textbf{11.0\spm{}0.0} & \textbf{11.0\spm{}0.0} & \textbf{11.0\spm{}0.0} &  1.0 &  1.0 &  1.0 \\\hline
    Zerocost(620) &  256 &  279 &  212 &  281 &  249.1\spm{}1.8 &  280.2\spm{}7.9 &  \textbf{287.2\spm{}2.4} &  280.2\spm{}4.2 &  264.9\spm{}1.8 &  \textbf{0.0} &  \textbf{.02} &  \textbf{0.0}  \\
  \hline                                    
    {\relsize{-1}airport-fuel(20)} &  \textbf{15} &  13 &  7 &  \textbf{15} &  14.2\spm{}0.9 &  13.8\spm{}0.6 &  14.4\spm{}0.7 &  10.4\spm{}0.5 &  14.4\spm{}0.7 &  .49 &  .06 &  1.0  \\
    {\relsize{-1}blocks-stack(20)} &  \textbf{17} &  \textbf{17} &  15 &  \textbf{17} &  \textbf{17.0\spm{}0.0} &  \textbf{17.1\spm{}0.3} &  \textbf{17.0\spm{}0.0} &  16.0\spm{}0.0 &  \textbf{17.0\spm{}0.0} &  1.0 &  .37 &  1.0  \\
%   {\relsize{-1}depot-fuel(22)} &  6 &  6 &  4 &  6 &  6.0\spm{}0.0 &  6.0\spm{}0.0 &  6.0\spm{}0.0 &  6.0\spm{}0.0 &  6.0\spm{}0.0 &  1.0 &  1.0 &  1.0  \\
%   {\relsize{-1}driverlog-fuel(20)} &  8 &  8 &  7 &  8 &  8.0\spm{}0.0 &  7.2\spm{}0.7 &  8.0\spm{}0.0 &  8.0\spm{}0.0 &  8.0\spm{}0.0 &  1.0 &  \textbf{.01} &  1.0  \\
    {\relsize{-1}elevators-up(20)} &  7 &  \textbf{13} &  7 &  \textbf{13} &  5.3\spm{}0.5 &  8.8\spm{}0.9 &  9.4\spm{}1.1 &  8.2\spm{}0.7 &  7.3\spm{}0.5 &  \textbf{0.0} &  .25 &  \textbf{0.0}  \\
%   {\relsize{-1}floortile-ink(20)} &  8 &  8 &  8 &  8 &  8.0\spm{}0.0 &  8.0\spm{}0.0 &  8.1\spm{}0.3 &  8.0\spm{}0.0 &  8.3\spm{}0.5 &  .37 &  .37 &  0.3  \\
    {\relsize{-1}freecell-move(20)} &  4 &  19 &  4 &  19 &  4.0\spm{}0.0 &  \textbf{19.4\spm{}0.5} &  16.5\spm{}0.7 &  16.6\spm{}0.8 &  5.0\spm{}0.4 &  \textbf{0.0} &  \textbf{0.0} &  \textbf{0.0}  \\
%   {\relsize{-1}grid-fuel(5)} &  1 &  1 &  1 &  1 &  1.0\spm{}0.0 &  1.0\spm{}0.0 &  1.0\spm{}0.0 &  1.0\spm{}0.0 &  1.0\spm{}0.0 &  1.0 &  1.0 &  1.0  \\
%   {\relsize{-1}gripper-move(20)} &  7 &  7 &  7 &  7 &  6.0\spm{}0.0 &  6.0\spm{}0.0 &  6.0\spm{}0.0 &  7.0\spm{}0.0 &  7.0\spm{}0.0 &  1.0 &  1.0 &  \textbf{0.0}  \\
%   {\relsize{-1}hiking-fuel(20)} &  9 &  9 &  8 &  9 &  9.0\spm{}0.0 &  9.0\spm{}0.0 &  9.0\spm{}0.0 &  9.0\spm{}0.0 &  9.0\spm{}0.0 &  1.0 &  1.0 &  1.0  \\
%   {\relsize{-1}logistics00-fuel(28)} &  16 &  16 &  15 &  16 &  15.0\spm{}0.0 &  15.0\spm{}0.0 &  15.0\spm{}0.0 &  16.0\spm{}0.0 &  16.0\spm{}0.0 &  1.0 &  1.0 &  \textbf{0.0}  \\
    {\relsize{-1}miconic-up(30)} &  16 &  17 &  10 &  17 &  15.4\spm{}0.5 &  18.0\spm{}1.3 &  19.8\spm{}1.0 &  \textbf{20.4\spm{}1.0} &  17.0\spm{}0.4 &  \textbf{0.0} &  \textbf{0.0} &  \textbf{0.0}  \\
    {\relsize{-1}mprime-succumb(35)} &  15 &  14 &  12 &  14 &  15.8\spm{}0.7 &  18.7\spm{}3.9 &  \textbf{20.1\spm{}0.7} &  18.6\spm{}2.0 &  17.9\spm{}0.5 &  \textbf{0.0} &  .23 &  \textbf{0.0}  \\
%   {\relsize{-1}mystery-feast(20)} &  7 &  5 &  5 &  5 &  7.2\spm{}0.4 &  6.2\spm{}0.7 &  7.2\spm{}0.4 &  7.2\spm{}0.7 &  7.3\spm{}0.5 &  1.0 &  \textbf{0.0} &  .65  \\
%   {\relsize{-1}nomystery-fuel(20)} &  10 &  10 &  9 &  10 &  10.0\spm{}0.0 &  10.0\spm{}0.0 &  10.0\spm{}0.0 &  9.4\spm{}0.5 &  10.0\spm{}0.0 &  1.0 &  1.0 &  1.0  \\
%   {\relsize{-1}parking-movecc(20)} &  0 &  0 &  0 &  0 &  0.0\spm{}0.0 &  0.0\spm{}0.0 &  0.0\spm{}0.0 &  0.0\spm{}0.0 &  0.0\spm{}0.0 &  1.0 &  1.0 &  1.0  \\
%   {\relsize{-1}pathways-fuel(30)} &  5 &  5 &  4 &  5 &  4.0\spm{}0.0 &  4.2\spm{}0.4 &  4.4\spm{}0.5 &  4.8\spm{}0.4 &  4.4\spm{}0.5 &  \textbf{.03} &  .37 &  1.0  \\
    {\relsize{-1}pipesnt-pushstart(20)} &  8 &  8 &  6 &  7 &  8.0\spm{}0.0 &  8.6\spm{}1.3 &  \textbf{9.8\spm{}0.4} &  \textbf{9.8\spm{}0.4} &  8.5\spm{}0.5 &  \textbf{0.0} &  \textbf{.04} &  \textbf{0.0}  \\
    {\relsize{-1}pipesworld-pushend(20)} &  3 &  4 &  2 &  4 &  3.0\spm{}0.0 &  4.2\spm{}1.0 &  4.5\spm{}0.8 &  \textbf{5.4\spm{}0.8} &  3.9\spm{}0.3 &  \textbf{0.0} &  0.5 &  \textbf{.05}  \\
%   {\relsize{-1}psr-small-open(20)} &  19 &  19 &  19 &  19 &  18.0\spm{}0.0 &  19.0\spm{}0.0 &  19.0\spm{}0.0 &  19.0\spm{}0.0 &  19.0\spm{}0.0 &  \textbf{0.0} &  1.0 &  1.0  \\
%   {\relsize{-1}rovers-fuel(40)} &  8 &  8 &  7 &  9 &  8.0\spm{}0.0 &  8.0\spm{}0.0 &  8.0\spm{}0.0 &  9.0\spm{}0.0 &  8.0\spm{}0.0 &  1.0 &  1.0 &  1.0  \\
    {\relsize{-1}scanalyzer-analyze(20)} &  9 &  9 &  3 &  9 &  \textbf{9.7\spm{}0.6} &  9.3\spm{}0.5 &  9.1\spm{}0.3 &  7.4\spm{}1.0 &  9.1\spm{}0.3 &  \textbf{.02} &  0.3 &  1.0  \\
%   {\relsize{-1}sokoban-pushgoal(20)} &  18 &  18 &  18 &  18 &  18.0\spm{}0.0 &  17.0\spm{}0.0 &  17.9\spm{}0.3 &  17.0\spm{}0.0 &  18.0\spm{}0.0 &  .37 &  \textbf{0.0} &  .37  \\
%   {\relsize{-1}storage-lift(20)} &  4 &  4 &  4 &  4 &  4.0\spm{}0.0 &  5.0\spm{}1.2 &  4.4\spm{}0.5 &  4.6\spm{}0.5 &  4.6\spm{}0.5 &  \textbf{.03} &  .26 &  .41  \\
%   {\relsize{-1}tidybot-motion(20)} &  16 &  16 &  14 &  16 &  16.0\spm{}0.0 &  16.0\spm{}0.0 &  16.0\spm{}0.0 &  15.6\spm{}0.5 &  16.0\spm{}0.0 &  1.0 &  1.0 &  1.0  \\
    {\relsize{-1}tpp-fuel(30)} &  8 &  \textbf{11} &  7 &  \textbf{11} &  7.0\spm{}0.0 &  \textbf{11.0\spm{}0.0} &  \textbf{11.0\spm{}0.0} &  \textbf{11.0\spm{}0.0} &  8.1\spm{}0.3 &  \textbf{0.0} &  1.0 &  \textbf{0.0}  \\
    {\relsize{-1}woodworking-cut(20)} &  5 &  7 &  2 &  7 &  4.5\spm{}0.5 &  6.7\spm{}0.5 &  \textbf{8.6\spm{}0.9} &  7.8\spm{}0.7 &  7.1\spm{}0.3 &  \textbf{0.0} &  \textbf{0.0} &  \textbf{0.0}  \\
%   {\relsize{-1}zenotravel-fuel(20)} &  7 &  7 &  7 &  7 &  7.0\spm{}0.0 &  7.0\spm{}0.0 &  7.0\spm{}0.0 &  7.0\spm{}0.0 &  7.0\spm{}0.0 &  1.0 &  1.0 &  1.0 \\
\hline
 Total(1724) &  814 &  844 &  654 &  837 &  803.7\spm{}2.2 &  848.5\spm{}8.9 &  \textbf{857.8\spm{}2.9} &  840.2\spm{}4.4 &  824.7\spm{}2.1 &  \textbf{0.0} &  \textbf{.01} &  \textbf{0.0} \\\hline
\end{tabular}

 \caption{
 Full version of the bottom line of \reftbl{depth} showing 
 the experiments on the IPC benchmark instances using \mands heuritics.
 Each cell shows the coverage of the domain solved with 5 min, 2GB.
 As in the original \reftbl{depth}, we highlighted the best results in
 \textbf{boldface} only when the maximum pairwise coverage difference $\mit{MaxDiff}>2$.
 }
 \label{mands-zerocost-full}
 }
\end{table*}



\end{document}
